\documentclass[a4j]{jarticle}
\date{}
\usepackage[dvipdfmx]{graphicx}

\begin{document}


%表題
\makeatletter %ここから\makeatotherまで触らなくていい
	\def\@thesis{平成28年度 東邦大学理学部情報科学科 卒業研究}
	\def\id#1{\def\@id{#1}}
	\def\department#1{\def\@department{#1}}
	
	\def\@maketitle{
		\begin{center}
			\vspace{10mm}
			{\large \@thesis \par}	%修士論文と記載される部分
			\vspace{50mm}
			{\huge\bf \@title \par}	% 論文のタイトル部分
			\vspace{15mm}
			{\Large 学籍番号 \@id \par}	% 学籍番号部分
			\vspace{5mm}
			{\Large \@author \par}	% 氏名
			\vspace{50mm}
		\end{center}
		\begin{flushright}
			{\large 金岡研究室}
		\end{flushright}
	}
\makeatother

\title{タイトルです} %中括弧の中に卒論のタイトル
\id{55000000} %学籍番号
\author{苗字と名前} %名前
\maketitle{\title} %表紙の出力
\thispagestyle{empty} %このページ(表紙)のページ番号を消去
\newpage %強制改ページ

\tableofcontents %目次の出力

\newpage
\section{はじめに} 
概要文みてかくといいかも

\newpage
\section{前提知識}


%%%%%%%%%%%%%%%%%%%Android%%%%%%%%%%%%%%%%%%%%
\subsection{Android}
Androidとは、Googel社が2007年に開発したスマートフォンやタブレット端末など携帯情報機器向けのOperating System、あるいはAndroid OSが搭載された端末を指す。またAndroid OSが搭載された端末のことである。主にスマートフォンのOSとして広く普及しており、世界的にApple社の携帯機器向けiOSと市場を二分している。


%%%%%%%%%%%%%%%%%%%Operating System%%%%%%%%%%%%%%%%%%%%
\subsection{Operating System}
Operating System(以後OS)とは、ソフトウェアの種類の1つで、機器の基本的な管理や制御のための機能や、多くのソフトウェアが共通して利用する基本的な機能などを実装したシステム全体を管理するソフトウェアのことである。


%%%%%%%%%%%%%%%%%%%アプリ%%%%%%%%%%%%%%%%%%%%
\subsection{アプリ}
アプリとは、Application Softwareの略であり、ゲームや音楽プレイヤー、メールなど、スマートフォンのOS上で動くソフトウェアのことを言う。


%%%%%%%%%%%%%%%%%%%Androidアプリケーション%%%%%%%%%%%%%%%%%%%%
\subsection{Androidアプリケーション}
Androidアプリケーションとは、Androidにインストールすることのできるアプリである。
主に、JavaやKotlinというプログラミング言語で作成されている。Javaプログラムをコンパイルして機械語に変換し、
画像などのリソースと合わせてAPKというパッケージにすることで インストール可能である。


%%%%%%%%%%%%%%%%%%%APK%%%%%%%%%%%%%%%%%%%%
\subsection{APK}
APKとは、Android Application Packageの略であり、Android向けのアプリケーションをAndroid端末にインストールできる形式にパッケージにしたもの、もしくはそのファイルのことである。
入手方法は000に後述するAPKストアからダウンロードする方法や、単体で公開されているAPKファイルをダウンロードする方法等が存在する。
一般的にAPKは“.apk”という拡張子を持つ。ただし、.apkファイル自体はzip形式で圧縮されており、その中にはアプリケーションの動作に必要なさまざまなファイルが納められている。
.apkファイルに対してzipファイルと同様の解凍処理を行い、得られるファイルのうち本研究に関連する項目を解説する。
\begin{itemize}
	\item  AndroidManifest.xml
		\begin{itemize}
			\item Androidアプリケーションの必要要件や、最初に起動されるアクティビティの記述がされている
			\item zipの解凍処理により得られるAndroifManifest.xmlはバイナリファイルの状態であるため、テキストエディタ等で内容を閲覧するためにはデコード処理が必要である
			\item デコードされたAndroidManifest.xmlの入手方法は後述する
		\end{itemize}
	\item classes.dex
		\begin{itemize}
			\item Android アプリケーションのソースファイルを変換してAndroidで実行できるようにしまとめたファイルである
			\item 1つのdexファイルに含められるメソッドの数は65,536が上限であり、それ以上の数のメソッドが1つのAndroidアプリケーションに含まれる場合は、classes2.dex、classes3.dex…と複数ファイルに分割される
		\end{itemize}
\end{itemize}


%%%%%%%%%%%%%%%%%%%バイナリファイル%%%%%%%%%%%%%%%%%%%%
\subsection{バイナリファイル}
バイナリファイルとは、コンピュータプログラムによって読み書きや処理を行うことを前提に、文字コードの規約を用いずに任意のビット列によって構成されるデータを格納するものである。
一方、テキストファイルは文字コードで規定された自然言語の文字と、表示制御のための少数の制御コード
%%(空白や改行など)
のみを含み、人間が容易に読み書きできる。
テキストファイルはテキストエディタなどで表示して中にどんな文字が書かれているかを読むことができるが、バイナリファイルはその形式に対応したソフトウェア以外ではまったく内容を知ることはできない。
ただし、バイナリエディタというソフトウェアによってどのようなバイト列が並んでいるかを見ることはできる。


%%%%%%%%%%%%%%%%%%%APKストア%%%%%%%%%%%%%%%%%%%%
\subsection{APKストア}
APKストアとは、Androidアプリケーション開発者の作成したAndroidアプリケーションの配信を代行するサービス、およびそれを行っているWebサイトのことである。Androidの公式APKストアは、Android の公式APKストアであるGooglePlay[ ]1つのみであり、非公式のAPKストアは数多く存在する。


%%%%%%%%%%%%%%%%%%%Android Developers%%%%%%%%%%%%%%%%%%%%
\subsection{Android Developers}
Android Developers とは、Android アプリケーション開発者向けのAndroid公式Webサイトのことである。Androidの詳細やドキュメントが提供されている。
公式ドキュメントといった場合Android Developersを指す。


%%%%%%%%%%%%%%%%%%%smaliファイル%%%%%%%%%%%%%%%%%%%%
\subsection{smaliファイル}
smaliとは、AndroidのDalvik仮想マシンで使用されるアセンブリ言語Smaliで書かれた開発者ファイルである。
通常、Androidアプリケーションに含まれている実行可能ファイルである。
DEX(Dalvik Executable)(Dalvik実行可能)ファイル(.apkファイル)を逆コンパイルすることによって作成される。
smaliファイルの取得には、Apktool[ ]を用いる方法と、baksmali[ ]を用いる方法がある。
それぞれのツールの詳細は000で説明する。


%%%%%%%%%%%%%%%%%%%中間言語%%%%%%%%%%%%%%%%%%%%
\subsection{中間言語}
中間言語とは、計算機が実行するコードを人間が理解できる形式で表現するための言語である。
以下に本研究に関連するDalvikバイトコードについての詳細な説明を述べる。


%%%%%%%%%%%%%%%%%%%Dalvikバイトコード%%%%%%%%%%%%%%%%%%%%
\subsection{Dalvikバイトコード}
Dalvikバイトコードとは、Androidにおける中間言語である。Apktool等を用いてAPKより取得できるsmaliファイルは、Dalvikバイトコードで記述されている。以下に、ソースコード 1、ソースコード2にDalvikバイトコードの例と、対応するソースコードを示す。

\lstinputlisting[caption=対応するソースコード,label=BytecodeSample.java]{C:/Users/arier/Documents/卒論/yamaguchi_thesis-main/コード/BytecodeSample.java}
\lstinputlisting[caption=Dalvikバイトコードの例,label=BytecodeSample.smali]{C:/Users/arier/Documents/卒論/yamaguchi_thesis-main/コード/BytecodeSample.smali}


%%%%%%%%%%%%%%%%%%%Linux%%%%%%%%%%%%%%%%%%%
\subsection{Linux}

Linuxとは、WindowsやmacOSといったOSの1つである。CUIベースのOSであり、コマンドを実行することでPCを操作することが可能である。


%%%%%%%%%%%%%%%%%%%CUI%%%%%%%%%%%%%%%%%%%
\subsection{CUI}
CUIとは、Character User Interfaceの略であり、コンピュータやソフトウェアが利用者に情報を提示したり操作を受け付けたりする方法の1つで、
すべてのやり取りを文字によって行う方式のことである。


%%%%%%%%%%%%%%%%%%%Ubuntu%%%%%%%%%%%%%%%%%%%
\subsection{Ubuntu}
UbuntuはLinux系のOSの1つである。
このアプリは、CUIでファイル操作が可能である点や、シェルスプリクトを利用してsmaliファイルの解析をおこなえるため、本研究で利用した。


%%%%%%%%%%%%%%%%%%%シェル%%%%%%%%%%%%%%%%%%%
\subsection{シェル}
シェルとは、「オペレーティングシステムと対話するためのインターフェイス」であり、コマンドなどを制御する「環境」のことである。
シェルがあることでコマンドを受付、OSとの対話ができるようになる。CUI環境においてシェルは最も身近なインターフェイスである。


%%%%%%%%%%%%%%%%%%%UNIXコマンド%%%%%%%%%%%%%%%%%%%
\subsection{UNIXコマンド}
UNIXコマンドとは、Linux OS等のUNIXマシンにおいてCUI上からコンピュータを操作するために使用するコマンドのことを言う。ファイルのコピーを行うcp、ファイルの内容を表示するcat、ディレクトリの内容を表示するlsなどが存在する。


%%%%%%%%%%%%%%%%%%%シェルスプリクト%%%%%%%%%%%%%%%%%%%%
\subsection{シェルスプリクト}
シェルスプリクトとは、OSを操作するためのシェル上で実行できる簡易なプログラム言語(スプリクト言語)のことを言う。また、スプリクト言語によって書かれた、複数のOSコマンドや制御文などを組み合わせたプログラムを指す。
shコマンドの引数としてシェルスクリプトのファイルを与えて実行すると、ファイルに記述されたUNIXコマンドが上から順に実行される。以下のシェルスクリプトを実行すると、a.txtがb.txtにコピーされ、a.txtの末尾に“hoge”という文字列が追加される。
\lstinputlisting[caption=シェルスクリプトの例,label=ShellScriptSample.sh]{C:/Users/arier/Documents/卒論/yamaguchi_thesis-main/コード/ShellScriptSample.sh}


%%%%%%%%%%%%%%%%%%%暗号技術%%%%%%%%%%%%%%%%%%%%
\subsection{暗号技術}

\subsubsection {MD5}
MD5とは、Message Digest algorithm 5の略であり、ハッシュ値を計算するためのハッシュ関数のひとつである。RSA暗号の開発者のひとり、ロン・リベスト氏らによって開発された。IPsecや、POP before SMTPなど、さまざまなセキュリティープロトコルで使われている一方、最近になって脆弱性も指摘されている。生成された値は「ハッシュ値」(hash value)と呼ばれる。

\subsubsection {SHA-1}
SHA-1とは、アメリカ国家安全保障局が考案し、1995年から米国政府の標準として使用されているハッシュ関数である。任意のデータから160ビットのハッシュ値を生成する。
2017年、GoogleがSHA-1でハッシュ値が衝突する事例[]を発見したため、より安全なハッシュ関数を使用することが推奨されている。

\subsubsection{SHA-2}
SHA-2とは、SHA-1を改良したハッシュ関数である。バリエーション豊富であり以下を総称してSHA-2と呼ばれている。
\begin{itemize}
\item SHA-224(ハッシュ値:224bit)
\item SHA-256(ハッシュ値:256bit)
\item SHA-384(ハッシュ値:384bit)
\item SHA-512(ハッシュ値:512bit)
\item SHA-512/224(ハッシュ値:224bit)
\item SHA-512/256(ハッシュ値:256bit)
\end{itemize}
ベースはSHA-256とSHA-512である。
SHA-224はSHA-256で出力されたハッシュ値を224bitに切り詰めたものであり、SHA-384はSHA-512で出力されたハッシュ値を384bitに切り詰めたものである。
SHA-512/224とSHA512/256についてもSHA-512で出力されたハッシュ値を224bit、256bitに切り詰めたものである。
大きな違いとしては、SHA-256は32bitCPU、SHA-512は64bitCPUに最適化されている点がある。
ハッシュ長が長い方がセキュリティ的な強度が高いが、負荷が高くなる。
ただし、現状SHA-256でも必要十分な強度となっているため、一般的にはSHA-256が利用されている。


%%%%%%%%%%%%%%%%%%%API%%%%%%%%%%%%%%%%%%%
\subsection{API}
APIとは、Application Programming Interfaceの略であり、あるコンピュータプログラム(ソフトウェア)の機能や管理するデータなどを、外部の他のプログラムから呼び出して利用するための手順やデータ形式などを定めた規約である。
これは、3種類に分けられる。

\subsubsection {公式API}
公式APIとは、Androidの開発者向け公式WebサイトであるAndroid DevelopersのAPIリファレンスに記載されているAPIである。

\subsubsection {サードパーティー製API}
サードパーティー製APIとは、サードパーティが提供するAPIのことである。
サードパーティとは、特定のハードウェア、OS、ソフトウェア、あるいはサービスなどを対象として、
それに対応する製品を販売、提供している組織や企業のことを指す。
Google社のtinkやFacebook社のConcealnがある。

\subsubsection {独自実装等のAPI}
API開発者が既存のAPIを利用せずに独自に実装したAPIや、先述2つに含まれないものを独自実装等のAPIと本論文では呼ぶこととする。


%%%%%%%%%%%%%%%%%%%APIドキュメント%%%%%%%%%%%%%%%%%%%
\subsection{APIドキュメント}
APIドキュメントとは、APIによる開発方法やクラス内のメソッドの使用方法を解説した説明書である。
APIリファレンスとも呼ばれる。






























\newpage
\section{関連研究} %先に書く
本研究における関連研究を紹介する。
\subsection{河合らの調査}
河合による調査は、Androidアプリケーションを調査対象とし、Androidアプリケーションの暗号技術利用に関する現状を明らかにするために、
401,971個のAPKより展開されたsmaliファイルと4,324個のAndroid DevelopersのAPIリファレンスに記載されているAPIより取得した暗号・セキュリティに関するクラスが持つメソッドのリストを使用し、
暗号で用いられるメソッド名や特徴のある用語によるフィルタリングアルゴリズムが指定可能な代表的箇所の抽出やAPIの利用傾向分析の調査を行った。
%メソッドリスト表いれるべき??


各APIにおける調査対象のAPK群の中でそのAPIを少なくとも1回は利用した回数とそのAPIが少なくとも1回は利用されている確率の上位10件を表\ref{tb:api}に示す。
調査対象の APK群に最も利用された数が多かったメソッドは
android.net.Uri.parse(java.lang.String) の248,145個
であり、次がjava.net.URL.URL(java.lang.String)の120,601個であった。


調査対象のAPK群に利用された数が多かったメソッドの上位はjava.netやandroid.netといったネットワークに関わるもの
であり、調査対象のAPK群のうち61.73\%のAPKがandroid.net.Uri.parse(java.lang.String)を使用しており、
何かしらの通信を行っていると考えられる。

android.net.Uri.parse(java.lang.String)を暗号・セキュリティに関する各APIにおける
調査対象のAPK群の中で
そのAPIを少なくとも1回は利用した回数と
そのAPIが少なくとも1回は利用されている確率の
上位10件を表\ref{tb:graph_cipher}
に示す。

暗号・セキュリティに関するメソッドとして、
java.security.MessageDigest.digest()の43,418個が最も利用されており、
次にjava.security.MessageDigest.getInstance(java.lang.String)の37,405個であった。


java.security.MessageDigestクラスは主にSHA-1やSHA-256といったアルゴリズムを使用したハッシュ値を提供するものである。
調査対象のAPK群のうち10.80\%がハッシュ値を利用していることがわかった。
また、
javax.crypto.spec.SecretKeySpec.SecretKeySpec(byte[],java.lang.String)表\ref{tb:graph_cipher}
でメッセージダイジェストに関するクラスの次に利用数が多い。
javax.crypto.spec.SecretKeySpecクラスは秘密鍵に関する機能を提供するクラスである。
調査対象のAPK群において最も利用されている暗号化方式は公開鍵暗号であることが考えられる使用しており、
何かしらの通信を行っていると考えられる。


\begin{table}[t]
\begin{center}
\caption{各APIにおける調査対象のAPK群の中でそのAPIを少なくとも1回は利用した回数とそのAPIが少なくとも1回は利用されている確率の上位10件}
\begin{tabular}{lrrr} \hline
メソッド名 & APK数(個) & 
\begin{tabular}{c}メソッド\\利用確率\\P(A)(%) \end{tabular} \\ \hline
~~android.net.Uri.parse(java.lang.String) & 248,145  & 61.73  \\
~~java.net.URL.URL(java.lang.String) & 120,601  & 30.00  \\
~~java.net.URL.openConnection() & 104,936  & 26.11  \\
\begin{tabular}{l}android.net.ConnectivityManager\\~~~~.getActiveNetworkInfo()\end{tabular} & 83,050  & 20.66  \\
~~android.net.Uri.fromFile(java.io.File) & 82,428  & 20.51  \\
\begin{tabular}{l}java.net.HttpURLConnection\\~~~~.getResponseCode()\end{tabular} & 63,770  & 15.86  \\
~~android.net.NetworkInfo.isConnected() & 61,452  & 15.29  \\
~~android.net.Uri.toString() & 58,250  & 14.49  \\
~~java.net.HttpURLConnection.disconnect() & 53,782  & 13.38  \\
\begin{tabular}{l}java.net.HttpURLConnection\\~~~~.setRequestMethod(java.lang.String)\end{tabular} & 52,515  & 13.06  \\ \hline
\end{tabular}
\label{tb:api}
\end{center}
\end{table}

\begin{table}[t]
\begin{center}
\caption{暗号・セキュリティに関する各APIにおける調査対象のAPK群の中でそのAPIを少なくとも1回は利用した回数とそのAPIが少なくとも1回は利用されている確率の上位10件}
\begin{tabular}{lrrr} \hline
メソッド名 & APK数(個) & 
\begin{tabular}{c}メソッド\\利用確率\\P(A)(%) \end{tabular} \\ \hline
~~java.security.MessageDigest.digest() & 43,418 & 10.80 \\
\begin{tabular}{l}java.security.MessageDigest\\~~~~.getInstance(java.lang.String)\end{tabular} & 37,405 & 9.31 \\
~~java.security.MessageDigest.reset() & 28,357 & 7.05 \\
~~java.security.MessageDigest.update(byte[],int,int) & 20,004 & 4.98\\
\begin{tabular}{l}javax.crypto.spec.SecretKeySpec\\~~~~.SecretKeySpec(byte[],java.lang.String)\end{tabular} & 15,829 & 3.94 \\
\begin{tabular}{l}java.security.KeyFactory\\~~~~.generatePublic(java.security.spec.KeySpec)\end{tabular} & 15,009 & 3.73 \\
\begin{tabular}{l}java.security.spec.X509EncodedKeySpec\\~~~~.X509EncodedKeySpec(byte[])\end{tabular} & 14,662 & 3.65 \\
~~java.security.SecureRandom.SecureRandom() & 13,759 & 3.42 \\
~~java.security.MessageDigest.digest(byte[]) & 13,414 & 3.34 \\
~~javax.crypto.Cipher.doFinal(byte[]) & 13,009 & 3.24 \\ \hline
\end{tabular}
\label{tb:graph_cipher}
\end{center}
\end{table}




河合による先行研究ではこの上記のような結果が得られたが、 この研究ではAndroid Developersに記載されている公式のAPIについての調査以外は
行われていないので、暗号利用動向の網羅的調査のために今後これらの更なる調査、分析が必要である。




































\newpage
\section{調査ターゲット(準備)}
\subsection {APKの取得方法}
\subsection{smaliファイル化方法}

\subsection{APK暗号技術}
 APKに使われている暗号技術は大きく分けて3種類ある。

 \subsubsection {公式API}
公式APIとは、Androidの開発者向け公式WebサイトであるAndroid DevelopersのAPIリファレンスに記載されているAPIである。

 \subsubsection {サードパーティー製API}
サードパーティー製APIとは、サードパーティが提供するAPIのことである。
サードパーティとは、特定のハードウェア、OS、ソフトウェア、あるいはサービスなどを対象として、
それに対応する製品を販売、提供している組織や企業のことを指す。
サードパーティ製APIの例としては、Google社のtinkやFacebook社のConcealnがある。

 \subsubsection {独自実装等のAPI}
API開発者が既存のAPIを利用せずに独自に実装したAPIや、先述2つに含まれないものを独自実装等のAPIと本論文では呼ぶこととする。

\subsection{   }
河合による先行研究では、Android Developers のAPI リファレンスに記載されているAPIから、暗号・セキュリティに関
するパッケージ、クラス、メソッドを抽出しリスト化した。
このリストをもとにAPKにおいてど
れほど暗号技術が利用されているかの分析を行った。

独自実装等のAPIはドキュメントが公開されている可能性
が低いためAPIのリスト化が困難である。これは、
RSAやECC、Cryptoといった暗号、セキュリティ
に関するキーワードをAPIのリストの代わりとし検
索する必要があるためAPKの網羅的調査を行う上
で困難である。

比較して、サードパーティー製API
ではドキュメントが公開されているものもあるので
リスト化の困難性が少ない。そこで、本研究では特
にサードパーティー製APIを本研究の分析対象とする。
サードパーティ製のAPIの分析ではまずAPIのリスト化を行う必要があるが、サードパーティー製
APIは公式APIとは違いドキュメントが作成され
ていないものがある。存在しない場合はサードパー
ティ製の APIのソースコードを解析し、APIのド
キュメントを作成してからAPIのリストの作成を行う。





\subsection{サードパーティー製APIの種類}
\subsubsection{Tink}

GoogleのTinkには4つのプリミティブがある。

    関連データを備えた認証付き暗号(プリミティブ: AEAD)

    メッセージ認証コード (プリミティブ: MAC)

    ディジタル署名(プリミティブ: PublicKeySignとPublicKeyVerify)

    ハイブリッド暗号化(プリミティブ: HybridEncryptとHybridDecrypt)


\subsection{APIの取得方法}
Tinkのドキュメントページ[ ]からAPIを抽出する。そのクラス(計167個)が持つ
メソッド計000個のリスト化を行った。このリストは、2020年00月のものである。
リストの1部を抜粋し、表0に示す。リスト全体は付録Aに示す。


\if
 APIの分析
-公式APIの分析
河合さんがやった
  -サードパーティ製APIの分析
   今回は、Google社のAPI、Tinkに対して研究を進めていく
  -独自実装等のAPIの分析
   残課題
\fi

\newpage
\section{調査手法}
あああああ

あああああ

\begin{itemize}
\item 調査方法
 -シェル作成内容の解説
  コマンド解説
\end{itemize}

\newpage
\section{調査結果と考察}

APIリストとsmaliファイルで利用されるメソッドのマッチング結果を下記に示す。\\
調査の結果、Tinkを利用しているのはパッケージ名が”mobi.zapzap”のAPK1つだけであった。
\\このAPKでは、AndroidKeysetManagerクラスとそのメソッドが使用されている。
このアプリケーション名は、”ZapZap - Mobile Wallet”\cite{ZapZap}であり日本では
サービスしていないAndroid版モバイルアプリケーションである。
\\使用されているAPIの分析から設定値を保存するSharedPreferencesへのアクセスにTink上のAndroidKeysetManagerクラスとそのメソッドを使用しているため、
暗号技術そのものとしてTinkは使用されていないと考えられる。
\\Tink利用がAndroid公式 API よりも大幅に少ない理由として、Tink1.0.0のリリース開始が2017年9月と最近である点と、
Androidでは公式APIの利用が中心的である点が考えられる。

\lstinputlisting[caption=APIリストとsmaliファイルで利用されるメソッドのマッチング結果,label=tink-result.txt]{C:/Users/arier/Documents/卒論/yamaguchi_thesis-main/コード/tink-result.txt}



\if0
\begin{landscape}
\begin{table}[t]
\begin{center}
\caption{ APIリストとsmaliファイルで利用されるメソッドのマッチング結果}
\scalefont{0.1}
\begin{tabular}{llrr}  \hline
smaliファイル名 & メソッド \\ \hline
172\_apks011\_smali/16CFF2C83B4B4550D446C4BD60890BA6C43B047FC2D9D81120EFFF05CE69884D/mobi/zapzap/utils/AppUtil.smali: & invoke-virtual \{p0\}, Lcom/google/crypto/tink/integration/android/AndroidKeysetManager\$Builder;-\textgreater build()Lcom/google/crypto/tink/integration/android/AndroidKeysetManager; \\
172\_apks011\_smali/16CFF2C83B4B4550D446C4BD60890BA6C43B047FC2D9D81120EFFF05CE69884D/mobi/zapzap/utils/AppUtil.smali: & invoke-virtual \{p0, v0\}, Lcom/google/crypto/tink/integration/android/AndroidKeysetManager\$Builder;-\textgreater withKeyTemplate(Lcom/google/crypto/tink/proto/KeyTemplate;)Lcom/google/crypto/tink/integration/android/AndroidKeysetManager\$Builder; \\
172\_apks011\_smali/16CFF2C83B4B4550D446C4BD60890BA6C43B047FC2D9D81120EFFF05CE69884D/mobi/zapzap/utils/AppUtil.smali: & invoke-virtual \{p0, v0\}, Lcom/google/crypto/tink/integration/android/AndroidKeysetManager\$Builder;-\textgreater withKeyTemplate(Lcom/google/crypto/tink/proto/KeyTemplate;)Lcom/google/crypto/tink/integration/android/AndroidKeysetManager\$Builder; \\
172\_apks011\_smali/16CFF2C83B4B4550D446C4BD60890BA6C43B047FC2D9D81120EFFF05CE69884D/mobi/zapzap/utils/AppUtil.smali: & invoke-virtual \{p0, v0\}, Lcom/google/crypto/tink/integration/android/AndroidKeysetManager\$Builder;-\textgreater withMasterKeyUri(Ljava/lang/String;)Lcom/google/crypto/tink/integration/android/AndroidKeysetManager\$Builder; \\
172\_apks011\_smali/16CFF2C83B4B4550D446C4BD60890BA6C43B047FC2D9D81120EFFF05CE69884D/mobi/zapzap/utils/AppUtil.smali: & invoke-virtual \{v0, p0, v1, v2\}, Lcom/google/crypto/tink/integration/android/AndroidKeysetManager\$Builder;-\textgreater withSharedPref(Landroid/content/Context;Ljava/lang/String;Ljava/lang/String;)Lcom/google/crypto/tink/integration/android/AndroidKeysetManager\$Builder; \\ 
172\_apks011\_smali/16CFF2C83B4B4550D446C4BD60890BA6C43B047FC2D9D81120EFFF05CE69884D/mobi/zapzap/utils/AppUtil.smali: & invoke-virtual \{p0\}, Lcom/google/crypto/tink/integration/android/AndroidKeysetManager;-\textgreater getKeysetHandle()Lcom/google/crypto/tink/KeysetHandle; \\ \hline
\end{tabular}
\label{tb:graph}
\end{center}
\end{table}
\end{landscape}
% \textgreater が >
% \textless が <
\fi

\newpage
\section{今後の課題}

 - Google以外のサードパーティー製APIの調査
 - 独自実装等のAPIの調査

\newpage
\begin{thebibliography}{99}

\bibitem{StatCounter Global Stats}Mobile Operating System Market Share Worldwide |  "StatCounter Global Stats",
http://gs.statcounter.com/os-market-share/mobile/, (参照 2021-01-21)

\bibitem{GooglePlay}
Google LLC, "Google Play", https://play.google.com/store, (参照 2021-01-21)

\bibitem{Android_Developers}
Android Developers, "Android Developers", https://developer.android.com/index.html?hl=ja, (参照 2021-01-21)

\bibitem{Apktool}
iBotPeaches, "Apktool", https://ibotpeaches.github.io/Apktool/, (参照 2021-01-21)

\bibitem{Baksmali}
JesusFreke, "Smali/baksmali", https://github.com/JesusFreke/smali, (参照 2021-01-21)


\bibitem{INTERNET Watch}
INTERNET Watch, "Google事例"https://internet.watch.impress.co.jp/docs/news/1046144.html, (参照 2021-01-21)

\bibitem{API_reference}
Android Developers, "API reference", https://developer.android.com/reference?hl=ja, (参照 2021-01-21)

\bibitem{Tink Cryptography}
Tink Cryptography API for Android,''Tink Cryptography API for Android'',https://google.github.io/tink/javadoc/tink-android/1.5.0/, (参照 2021-01-21)

\bibitem{bib03}
だれだれ, "文献3", 年度
\bibitem{bib04}
だれだれ, "文献4", 年度
\bibitem{bib05}
だれだれ, "文献5", 年度
\bibitem{bib06}
だれだれ, "文献6", 年度
\bibitem{bib07}
だれだれ, "文献7", 年度
\bibitem{bib08}
だれだれ, "文献8", 年度
\bibitem{bib09}
だれだれ, "文献9", 年度
\bibitem{bib010}
だれだれ, "文献10", 年度
\bibitem{bib010}
だれだれ, "文献10", 年度
\bibitem{bib010}
だれだれ, "文献10", 年度
\bibitem{bib010}
だれだれ, "文献10", 年度
\bibitem{bib010}
だれだれ, "文献10", 年度
\bibitem{bib010}
だれだれ, "文献10", 年度
\bibitem{bib010}
だれだれ, "文献10", 年度
\bibitem{bib010}
だれだれ, "文献10", 年度
\bibitem{bib010}
だれだれ, "文献10", 年度
\bibitem{bib010}
だれだれ, "文献10", 年度
\bibitem{bib010}
だれだれ, "文献10", 年度
\bibitem{bib010}
だれだれ, "文献10", 年度
\bibitem{bib010}
だれだれ, "文献10", 年度

\end{thebibliography}

\end{document}