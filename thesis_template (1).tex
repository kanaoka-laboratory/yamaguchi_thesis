\documentclass[a4j]{jarticle}
\date{}
\usepackage[dvipdfmx]{graphicx}
\usepackage{longtable}
\usepackage{lscape}
\usepackage{color}
\usepackage{scalefnt}
\usepackage{listings}		% ソースコード挿入を使うおまじない
\bibliographystyle{junsrt}			% 参考文献を別ファイルにまとめてますよ
\lstset{	% ソースコード埋め込みの設定
	basicstyle={\small},%ソースコードの文字を小さくする
	commentstyle={\small\itshape},%コメントアウトの文字を小さくする
	frame=single,	% 枠線は単線
	tabsize=4,		% タブのサイズは4
	breaklines=true	% 自動改行
}
\begin{document}


%表題
\makeatletter %ここから\makeatotherまで触らなくていい
	\def\@thesis{令和2年度 東邦大学理学部情報科学科 卒業研究}
	\def\id#1{\def\@id{#1}}
	\def\department#1{\def\@department{#1}}
	
	\def\@maketitle{
		\begin{center}
			\vspace{10mm}
			{\large \@thesis \par}	%修士論文と記載される部分
			\vspace{50mm}
			{\huge\bf \@title \par}	% 論文のタイトル部分
			\vspace{15mm}
			{\Large 学籍番号 \@id \par}	% 学籍番号部分
			\vspace{5mm}
			{\Large \@author \par}	% 氏名
			\vspace{50mm}
		\end{center}
		\begin{flushright}
			{\large 金岡研究室}
		\end{flushright}
	}
\makeatother

\title{Androidアプリケーションにおけるサードパーティー製APIでの暗号技術利用動向の調査} %中括弧の中に卒論のタイトル
\id{5517097} %学籍番号
\author{山口千尋} %名前
\maketitle{\title} %表紙の出力
\thispagestyle{empty} %このページ(表紙)のページ番号を消去
\newpage %強制改ページ

\tableofcontents %目次の出力

\newpage
\section{はじめに} 

\newpage
\section{前提知識}

\subsection{Android}
Androidとは、Googel社が2007年に開発した、スマートフォンやタブレット端末など携帯情報機器向けのOperating Systemのこと。主にスマートフォンのOSとして広く普及しており、世界的にApple社の携帯機器向けiOSと市場を2分している。

\subsection{Operating System}
Operating System(以後OS)とは、ソフトウェアの種類の一つで、機器の基本的な管理や制御のための機能や、多くのソフトウェアが共通して利用する基本的な機能などを実装した、システム全体を管理するソフトウェアのこと。

\subsection{Androidアプリケーション}
Javaというプログラミンング言語で作成されている。
Javaプログラムをコンパイルして機械語に変換し、画像などのリソースと合わせてapkというパッケージにすることでAndroidにインストールすることのできるアプリになります。

\subsection{APK}
Android Application Packageの略であり、Android向けのアプリケーションをAndroid端末にインストールできる形式にパッケージにしたもの、もしくはそのファイルのこと。APKとは「.apk」という拡張子を持つAndroidアプリケーションの本体ファイルのことである。
ただし、apkファイル自体はZIP形式で圧縮されており、その中にはアプリケーションの動作に必要なさまざまなファイルが納められている。
apkのファイル形式でなければAndroid端末にアプリケーションをインストールすることができないため、Google Playなどで配信されているアプリケーションは、すべてapkファイルとして公開されている。

\subsetion{smaliファイル}
Androidのapkをapktoolやbaksmaliを用いて逆コンパイルするとsmaliファイルがでてくる。
\subsection{暗号API}
\begin{itemize}
\item MD5
MD5(Message Digest algorithm 5)
MD5は、ハッシュ値を計算するためのハッシュ関数のひとつで、RSA暗号の開発者のひとり、ロン・リベスト氏らによって開発された。
IPsecや、POP before SMTPなど、さまざまなセキュリティープロトコルで使われている一方、最近になって脆弱性も指摘されている。

\item SHA-1
SHA-1(シャーワン)とは Secure Hash Algorithm 1の略で、入力データを一定の手順で計算を行い、
入力値のデータの長さに関わらず決まった長さの文字列を出力するハッシュ関数の一つ。生成された値は「ハッシュ値」(hash value)と呼ばれる。
SHA-1はNSA(米国家安全保障局)が考案し、1995年にNIST(米国標準技術局)によって連邦情報処理標準の一つ(FIPS 180-1)として標準化された。
2005年頃から効率的に攻撃する手法がいくつか発見され十分な安全性が保たれなくなったため、近年では2001年に制定された後継のSHA-2規格への移行が進んでいる。

\item SHA-2
SHA-2はハッシュ関数の計算手順(アルゴリズム)を定義しており、どんな長さのデータからも常に同じ長さのハッシュ値を生成する。
同じ原文からは必ず同じ値が得られる一方、少しでも異なる原文からはまったく違う値が得られる。
データの伝送や複製を行なう際に、入力側と出力側でハッシュ値を求め一致すれば、途中で改ざんや欠落などが起こっていないことを確認することができる。
また、暗号や認証、デジタル署名などの要素技術として様々な場面で利用されている。 
\end{itemize}

\subsection{API}
APIとはApplication Programming Interfaceの略であり、あるコンピュータプログラム(ソフトウェア)の機能や管理するデータなどを、
外部の他のプログラムから呼び出して利用するための手順やデータ形式などを定めた規約のこと。
\begin{itemize}

\item 公式API
Androidの開発者向け公式WebサイトであるAndroid DevelopersのAPIリファレンスに記載されているAPIのこと。

\item サードパーティー製API
サードパーティーとは、特定のハードウェア、OS、ソフトウェア、あるいはサービスなどを対象として、それに対応する(プラットフォーム上で動作する、もしくは互換性のある)製品を販売・提供していること。
企業が提供しているものや, 開発者が提供しているものがあるサードパーティー製APIという。
Google社のtink(ティンク)やFacebook社のConcealn(コンシール)がある。


\item 独自実装等のAPI
API開発者が既存のAPIを利用せずに独自に実装したAPIや,先述2つに含まれないものを独自実装等のAPIとします.

\end{itemize}

\subsection{Android Developers}
公式ドキュメントといった場合Android Developersを指す。
Android Developers とは,Android アプリケーション開発者向けのAndroid公式Webサイトのこと。Androidの詳細やドキュメントが提供されている

\subsection{APIドキュメント}
APIリファレンスとも呼ばれる。
APIによる開発方法やクラス内のメソッドの使用方法を解説した説明書のこと

\subsection{Liuxとは}
LinuxとはOSの一種で、パソコンを動かすのに必要な基本ソフトウェアの1つ。

\subsection{Ubuntu}
UbuntuはLinux系のOS。

\subsection{シェル}
コンピュータのOSを構成するソフトウェアの1つで、利用者からの操作の受付や、利用者への情報の提示などを担当するもの。

\subsection{シェルスプリクト}
OSを操作するためのシェル上で実行(スプリクト言語)。また、そのような言語によって書かれた、複数のOSコマンドや制御文などを組み合わせた簡易なプログラム。一般的にはUNIX系OSのシェルで実行できるものを指す



\newpage
\section{関連研究} %先に書く
本研究における関連研究を紹介する。
\subsection{河合らの調査}
河合による調査は、Androidアプリケーションを調査対象とし、Androidアプリケーションの暗号技術利用に関する現状を明らかにするために暗号で用いられるメソッド名や特徴のある用語によるフィルタリングアルゴリズムが指定可能な代表的箇所の抽出やAPIの利用傾向分析をしていた。ていた。しかし、河合の研究では公式APIのみの調査しか行われていない。


\subsection{Yに関連した研究}
あああ
\begin{itemize}
\item まずは
\item この章のなかで書くことを
\item 箇条書きで書き出してみる
\item ことから始めましょう
\end{itemize}

\newpage

\section{調査対象と手法}
%%%%%%%%%%%%%%%%%%%%
\subsection {APKの取得}

APKにおいて利用されるAPIの調査対象としてAndroZooのデータセットよりAPKを411,486個用意し、APKをダウンロードした。
APKからsmaliファイルを展開する方法として、Apktoolを使用していた。
本研究でも、同じAPKを使用する。


AndroZooの参考文献

%%%%%%%%%%%%%%%%%%%%
\subsection{Android APIの分類}
APKに使われている暗号技術は大きく分けて3種類ある。

%%%%%%%%%%
 \subsubsection {公式API}

\if0
・Androidでは、ソフトウェア開発のために必要なプログラムやライブラリをGoogle社がAndroid SDKとして提供してる
・SDKで提供されるライブラリはAndroid開発者向けサイトAndroid DevelopersにAPIリファレンスとして記載されてる
・この研究ではここに記載されてるAPIを「公式API」と呼ぶことにするよ
\fi
Androidの開発者向け公式WebサイトであるAndroid DevelopersのAPIリファレンスに記載されているAPIのことを本研究では公式APIと呼ぶこととする。

%%%%%%%%%%
 \subsubsection {サードパーティー製API}
サードパーティー製APIとは、サードパーティが提供するAPIのことである。
サードパーティとは、特定のハードウェア、OS、ソフトウェア、あるいはサービスなどを対象として、
それに対応する製品を販売、提供している組織や企業のことを指す。
%if0
・AndroidにもサードパーティAPIが多くあるよ。
%fi

%%%%%%%%%%
 \subsubsection {独自実装等のAPI}
API開発者が既存のAPIを利用せずに独自に実装したAPIや、先述2つに含まれないものを独自実装等のAPIと本論文では呼ぶこととする。

%%%%%%%%%%%%%%%%%%%%
\subsection{APIの分析}
河合による先行研究では、公式APIから、暗号・セキュリティに関
するパッケージ、クラス、メソッドを抽出しリスト化を行った。
このリストをもとにAPKにおいてど
れほど暗号技術が利用されているかの分析を行った。

独自実装等のAPIはドキュメントが公開されている可能性
が低いためAPIのリスト化が困難である。これは、
RSAやECC、Cryptoといった暗号、セキュリティ
に関するキーワードをAPIのリストの代わりとし検
索する必要があるためAPKの網羅的調査を行う上
で困難である。

比較して、サードパーティー製API
ではドキュメントが公開されているものもあるので
リスト化の困難性が少ない。
サードパーティ製のAPIの分析ではまずAPIのリスト化を行う必要があるが、サードパーティー製
APIは公式APIとは違いドキュメントが作成され
ていないものがある。存在しない場合はサードパー
ティ製の APIのソースコードを解析し、APIのド
キュメントを作成してからAPIのリストの作成を行う。

独自実装等のAPIは今後の課題とする。

%%%%%%%%%%%%%%%%%%%%
\subsection{サードパーティ製APIの分析}

%%%%%%%%%%
\subsubsection{サードパーティ製API例}
どんなサードパーティ製APIが存在するのか紹介する。
\begin{itemize}
\item Tink 

Tinkは、Googleの暗号技術者とセキュリティエンジニアのグループが開発した、
多言語でクロスプラットフォームな暗号ライブラリである。

\item Conceal

Concealは、Facebookが開発したライブラリである。共通鍵暗号アルゴリズム AES(256bit)と暗号利用モードGCMを用いた暗号化処理を代行している。

\item Geduldig 

 
Geduldig は、Twitterが開発した
\end{itemize}
Tinkは、Android OSを提供しているGoogle社によるサードパーティ製APIであるため、Androidアプリケーション開発者にも利用されている可能性は高いと
考えられるため本研究の調査対象とする。


%%%%%%%%%%
\subsubsection{Tinkの分析}

Tinkは現在、それぞれのプリミティブを使って実装された、4つの暗号化操作を提供している。
\begin{itemize}
\item 関連データを備えた認証付き暗号(プリミティブ: AEAD)
\item メッセージ認証コード (プリミティブ: MAC)
\item ディジタル署名(プリミティブ: PublicKeySignとPublicKeyVerify)
\item ハイブリッド暗号化(プリミティブ: HybridEncryptとHybridDecrypt)
\end{itemize}
プリミティブとは、単純あるいは基本的な構造や要素のことを言う。
Tinkには、ドキュメントが存在するので、ドキュメントが存在しない
サードパーティ製APIよりリスト化の困難性が少ない。


\subsubsection{それぞれのプリミティブの詳しい説明いれるか迷っている}
暗号の説明をしていく。
\begin{itemize}
\item AEAD

AEADは、次の 3 つのアルゴリズムの組 AE = (AE-K, AE-E, AE-D) で定義される。

\item MAC

メッセージ認証コードMACは、Message Authentication Codeの略であり、ネットワークを通じて伝送されたメッセージが途中で改竄されていないかを確認することである。

\item PublicKeySignとPublicKeyVerify

\item HybridEncryptとHybridDecrypt

\end{itemize}
%%%%%%%%%%%%%%%%%%%%
\subsection{APIの取得方法}
Tinkのドキュメント\cite{Tink Cryptography}からAPIを抽出する。そのクラス(計167個)が持つ
メソッド計000個のリスト化を行った。このリストは、2020年00月のものである。
リストの1部を抜粋し、表0に示す。リスト全体は付録Aに示す。




%%\newpage
\section{提案手法の試作みたいなのを書く部分}
あああああ

あああああ

\begin{itemize}
\item まずは
\item この章のなかで書くことを
\item 箇条書きで書き出してみる
\item ことから始めましょう
\end{itemize}

\newpage
\section{調査結果と考察}

APIリストとsmaliファイルで利用されるメソッドのマッチング結果を下記に示す。\\
調査の結果、Tinkを利用しているのはパッケージ名が”mobi.zapzap”のAPK1つだけであった。
\\このAPKでは、AndroidKeysetManagerクラスとそのメソッドが使用されている。
このアプリケーション名は、”ZapZap - Mobile Wallet”\cite{ZapZap}であり日本では
サービスしていないAndroid版モバイルアプリケーションである。
\\使用されているAPIの分析から設定値を保存するSharedPreferencesへのアクセスにTink上のAndroidKeysetManagerクラスとそのメソッドを使用しているため、
暗号技術そのものとしてTinkは使用されていないと考えられる。
\\Tink利用がAndroid公式 API よりも大幅に少ない理由として、Tink1.0.0のリリース開始が2017年9月と最近である点と、
Androidでは公式APIの利用が中心的である点が考えられる。

\lstinputlisting[caption=APIリストとsmaliファイルで利用されるメソッドのマッチング結果,label=tink-result.txt]{C:/Users/arier/Documents/卒論/yamaguchi_thesis-main/コード/tink-result.txt}



\if0
\begin{landscape}
\begin{table}[t]
\begin{center}
\caption{ APIリストとsmaliファイルで利用されるメソッドのマッチング結果}
\scalefont{0.1}
\begin{tabular}{llrr}  \hline
smaliファイル名 & メソッド \\ \hline
172\_apks011\_smali/16CFF2C83B4B4550D446C4BD60890BA6C43B047FC2D9D81120EFFF05CE69884D/mobi/zapzap/utils/AppUtil.smali: & invoke-virtual \{p0\}, Lcom/google/crypto/tink/integration/android/AndroidKeysetManager\$Builder;-\textgreater build()Lcom/google/crypto/tink/integration/android/AndroidKeysetManager; \\
172\_apks011\_smali/16CFF2C83B4B4550D446C4BD60890BA6C43B047FC2D9D81120EFFF05CE69884D/mobi/zapzap/utils/AppUtil.smali: & invoke-virtual \{p0, v0\}, Lcom/google/crypto/tink/integration/android/AndroidKeysetManager\$Builder;-\textgreater withKeyTemplate(Lcom/google/crypto/tink/proto/KeyTemplate;)Lcom/google/crypto/tink/integration/android/AndroidKeysetManager\$Builder; \\
172\_apks011\_smali/16CFF2C83B4B4550D446C4BD60890BA6C43B047FC2D9D81120EFFF05CE69884D/mobi/zapzap/utils/AppUtil.smali: & invoke-virtual \{p0, v0\}, Lcom/google/crypto/tink/integration/android/AndroidKeysetManager\$Builder;-\textgreater withKeyTemplate(Lcom/google/crypto/tink/proto/KeyTemplate;)Lcom/google/crypto/tink/integration/android/AndroidKeysetManager\$Builder; \\
172\_apks011\_smali/16CFF2C83B4B4550D446C4BD60890BA6C43B047FC2D9D81120EFFF05CE69884D/mobi/zapzap/utils/AppUtil.smali: & invoke-virtual \{p0, v0\}, Lcom/google/crypto/tink/integration/android/AndroidKeysetManager\$Builder;-\textgreater withMasterKeyUri(Ljava/lang/String;)Lcom/google/crypto/tink/integration/android/AndroidKeysetManager\$Builder; \\
172\_apks011\_smali/16CFF2C83B4B4550D446C4BD60890BA6C43B047FC2D9D81120EFFF05CE69884D/mobi/zapzap/utils/AppUtil.smali: & invoke-virtual \{v0, p0, v1, v2\}, Lcom/google/crypto/tink/integration/android/AndroidKeysetManager\$Builder;-\textgreater withSharedPref(Landroid/content/Context;Ljava/lang/String;Ljava/lang/String;)Lcom/google/crypto/tink/integration/android/AndroidKeysetManager\$Builder; \\ 
172\_apks011\_smali/16CFF2C83B4B4550D446C4BD60890BA6C43B047FC2D9D81120EFFF05CE69884D/mobi/zapzap/utils/AppUtil.smali: & invoke-virtual \{p0\}, Lcom/google/crypto/tink/integration/android/AndroidKeysetManager;-\textgreater getKeysetHandle()Lcom/google/crypto/tink/KeysetHandle; \\ \hline
\end{tabular}
\label{tb:graph}
\end{center}
\end{table}
\end{landscape}
% \textgreater が >
% \textless が <
\fi

%%%%%%%%%%%%%%%%%%%%今後の課題%%%%%%%%%%%%%%%%%%%%
\newpage
\section{今後の課題}
%%%%%%%%%%%%%%%%%%%%
\subsection{他のサードパーティ製APIの調査}
\ref{sec:third_party}	で述べた通り、サードパーティ製APIはTink以外も存在するので、調査の幅を広げることが可能であると考える。また、他のサードパーティ製APIには\ref{sec:APIの分析}で述べた通り、公式APIとは違い、ドキュメントが公開されていないものもある。サードパーティ製のAPIの分析ではまず、そのAPIについてのドキュメントが存在しているかの確認を行い、存在する場合はそのドキュメントからAPIのリストを作成する。
存在しない場合はサードパーティ製のAPIのソースコードを解析し、APIのドキュメントを作成してからAPIのリストの作成を行う。
APKの解析の前にAPIの解析を行う必要があるという点が公式APIとの違いである。
作成したリストをもとに\ref{sec:調査}と同様の調査を行うことが可能であると考える。
%%%%%%%%%%%%%%%%%%%%
\subsection{独自実装等のAPIの調査}
\ref{sec:APIの分析}で述べた通り、独自実装等のAPIは、ドキュメントが作成され、かつ公開されている可能性が低いため、\ref{sec:APIの取得方法}の様にAPIのドキュメントからAPIのリストを作成することが不可能である。
APIのリストではなく開発者が独自実装等の際に利用する可能性の高いキーワード等を調査、整理しsmaliファイル内で調査する。
そしてその調査対象のキーワード群がどのメソッド名や引数として利用されているのかを調査しさらにそれらを利用しているメソッドやクラスを発見することで独自実装等のAPIを調査することが可能であると考える。

\newpage
\section{まとめ}
まとめええええええええええええええええええ

\newpage
\begin{thebibliography}{99}

\bibitem{StatCounter Global Stats}Mobile Operating System Market Share Worldwide |  "StatCounter Global Stats",
http://gs.statcounter.com/os-market-share/mobile/, (参照 2021-01-21)

\bibitem{GooglePlay}
Google LLC, "Google Play", https://play.google.com/store, (参照 2021-01-21)

\bibitem{Android_Developers}
Android Developers, "Android Developers", https://developer.android.com/index.html?hl=ja, (参照 2021-01-21)

\bibitem{Apktool}
iBotPeaches, "Apktool", https://ibotpeaches.github.io/Apktool/, (参照 2021-01-21)

\bibitem{Baksmali}
JesusFreke, "Smali/baksmali", https://github.com/JesusFreke/smali, (参照 2021-01-21)


\bibitem{INTERNET Watch}
INTERNET Watch, "Google事例"https://internet.watch.impress.co.jp/docs/news/1046144.html, (参照 2021-01-21)

\bibitem{API_reference}
Android Developers, "API reference", https://developer.android.com/reference?hl=ja, (参照 2021-01-21)

\bibitem{Tink Cryptography}
Tink Cryptography API for Android,''Tink Cryptography API for Android'',https://google.github.io/tink/javadoc/tink-android/1.5.0/, (参照 2021-01-21)



\bibitem{bib03}
だれだれ, "文献3", 年度
\bibitem{bib04}
だれだれ, "文献4", 年度
\bibitem{bib05}
だれだれ, "文献5", 年度
\bibitem{bib06}
だれだれ, "文献6", 年度
\bibitem{bib07}
だれだれ, "文献7", 年度
\bibitem{bib08}
だれだれ, "文献8", 年度
\bibitem{bib09}
だれだれ, "文献9", 年度
\bibitem{bib010}
だれだれ, "文献10", 年度
\bibitem{bib010}
だれだれ, "文献10", 年度
\bibitem{bib010}
だれだれ, "文献10", 年度
\bibitem{bib010}
だれだれ, "文献10", 年度
\bibitem{bib010}
だれだれ, "文献10", 年度
\bibitem{bib010}
だれだれ, "文献10", 年度
\bibitem{bib010}
だれだれ, "文献10", 年度
\bibitem{bib010}
だれだれ, "文献10", 年度
\bibitem{bib010}
だれだれ, "文献10", 年度
\bibitem{bib010}
だれだれ, "文献10", 年度
\bibitem{bib010}
だれだれ, "文献10", 年度
\bibitem{bib010}
だれだれ, "文献10", 年度
\bibitem{bib010}
だれだれ, "文献10", 年度

\end{thebibliography}

%%%%%%%%%%%%%%%%%%%%付録%%%%%%%%%%%%%%%%%%%%
\appendix
\section{Android DevelopersのAPIリファレンスに記載されているメソッドを抽出して作成したリスト}
\label{tab:HTMLList}
\begingroup
\scalefont{0.5}
\begin{landscape}
\begin{longtable}{lp{160mm}}
  \\
  

  %------ 最初のページの表の最上部 ----
  \hline
  クラス名 & メソッド名(引数) \\ \hline
  \endfirsthead
  %------ 2ページ以降の表の最上部 ----
  \multicolumn{2}{r}{前ページからの続き} \\ \hline
  クラス名 & メソッド名(引数) \\ \hline
  \endhead
  %----- ページの表の最下部 --------
  \hline
  \multicolumn{2}{r}{次ページに続く} \\
  \endfoot
  %----- 最終ページの表の最下部 --------
  \hline
  \multicolumn{2}{r}{以上} \\
  \endlastfoot
  
ObjectName & MethodName \\ 
com/google/crypto/tink/Aead	&	decrypt	\\
com/google/crypto/tink/Aead	&	encrypt	\\
com/google/crypto/tink/aead/AeadConfig	&	init	\\
com/google/crypto/tink/aead/AeadConfig	&	register	\\
com/google/crypto/tink/aead/AeadConfig	&	registerStandardKeyTypes	\\
com/google/crypto/tink/aead/subtle/AeadFactory	&	createAead	\\
com/google/crypto/tink/aead/subtle/AeadFactory	&	getKeySizeInBytes	\\
com/google/crypto/tink/aead/AeadKeyTemplates	&	createAesCtrHmacAeadKeyTemplate	\\
com/google/crypto/tink/aead/AeadKeyTemplates	&	createAesEaxKeyTemplate	\\
com/google/crypto/tink/aead/AeadKeyTemplates	&	createAesGcmKeyTemplate	\\
com/google/crypto/tink/aead/AeadKeyTemplates	&	createKmsAeadKeyTemplate	\\
com/google/crypto/tink/aead/AeadKeyTemplates	&	createKmsEnvelopeAeadKeyTemplate	\\
com/google/crypto/tink/aead/AeadWrapper	&	getInputPrimitiveClass	\\
com/google/crypto/tink/aead/AeadWrapper	&	getPrimitiveClass	\\
com/google/crypto/tink/aead/AeadWrapper	&	register	\\
com/google/crypto/tink/aead/AeadWrapper	&	wrap	\\
com/google/crypto/tink/mac/AesCmacKeyManager	&	aes256CmacTemplate	\\
com/google/crypto/tink/mac/AesCmacKeyManager	&	getKeyType	\\
com/google/crypto/tink/mac/AesCmacKeyManager	&	getVersion	\\
com/google/crypto/tink/mac/AesCmacKeyManager	&	keyFactory	\\
com/google/crypto/tink/mac/AesCmacKeyManager	&	keyMaterialType	\\
com/google/crypto/tink/mac/AesCmacKeyManager	&	parseKey	\\
com/google/crypto/tink/mac/AesCmacKeyManager	&	rawAes256CmacTemplate	\\
com/google/crypto/tink/mac/AesCmacKeyManager	&	register	\\
com/google/crypto/tink/mac/AesCmacKeyManager	&	validateKey	\\
com/google/crypto/tink/prf/AesCmacPrfKeyManager	&	aes256CmacTemplate	\\
com/google/crypto/tink/prf/AesCmacPrfKeyManager	&	getKeyType	\\
com/google/crypto/tink/prf/AesCmacPrfKeyManager	&	getVersion	\\
com/google/crypto/tink/prf/AesCmacPrfKeyManager	&	keyFactory	\\
com/google/crypto/tink/prf/AesCmacPrfKeyManager	&	keyMaterialType	\\
com/google/crypto/tink/prf/AesCmacPrfKeyManager	&	parseKey	\\
com/google/crypto/tink/prf/AesCmacPrfKeyManager	&	register	\\
com/google/crypto/tink/prf/AesCmacPrfKeyManager	&	validateKey	\\
com/google/crypto/tink/aead/AesCtrHmacAeadKeyManager	&	aes128CtrHmacSha256Template	\\
com/google/crypto/tink/aead/AesCtrHmacAeadKeyManager	&	aes256CtrHmacSha256Template	\\
com/google/crypto/tink/aead/AesCtrHmacAeadKeyManager	&	getKeyType	\\
com/google/crypto/tink/aead/AesCtrHmacAeadKeyManager	&	getVersion	\\
com/google/crypto/tink/aead/AesCtrHmacAeadKeyManager	&	keyFactory	\\
com/google/crypto/tink/aead/AesCtrHmacAeadKeyManager	&	keyMaterialType	\\
com/google/crypto/tink/aead/AesCtrHmacAeadKeyManager	&	parseKey	\\
com/google/crypto/tink/aead/AesCtrHmacAeadKeyManager	&	register	\\
com/google/crypto/tink/aead/AesCtrHmacAeadKeyManager	&	validateKey	\\
com/google/crypto/tink/subtle/AesCtrHmacStreaming	&	expectedCiphertextSize	\\
com/google/crypto/tink/subtle/AesCtrHmacStreaming	&	getCiphertextOffset	\\
com/google/crypto/tink/subtle/AesCtrHmacStreaming	&	getCiphertextOverhead	\\
com/google/crypto/tink/subtle/AesCtrHmacStreaming	&	getCiphertextSegmentSize	\\
com/google/crypto/tink/subtle/AesCtrHmacStreaming	&	getFirstSegmentOffset	\\
com/google/crypto/tink/subtle/AesCtrHmacStreaming	&	getHeaderLength	\\
com/google/crypto/tink/subtle/AesCtrHmacStreaming	&	getPlaintextSegmentSize	\\
com/google/crypto/tink/subtle/AesCtrHmacStreaming	&	newDecryptingChannel	\\
com/google/crypto/tink/subtle/AesCtrHmacStreaming	&	newDecryptingStream	\\
com/google/crypto/tink/subtle/AesCtrHmacStreaming	&	newEncryptingChannel	\\
com/google/crypto/tink/subtle/AesCtrHmacStreaming	&	newEncryptingStream	\\
com/google/crypto/tink/subtle/AesCtrHmacStreaming	&	newSeekableDecryptingChannel	\\
com/google/crypto/tink/subtle/AesCtrHmacStreaming	&	newStreamSegmentDecrypter	\\
com/google/crypto/tink/subtle/AesCtrHmacStreaming	&	newStreamSegmentEncrypter	\\
com/google/crypto/tink/streamingaead/AesCtrHmacStreamingKeyManager	&	aes128CtrHmacSha2561MBTemplate	\\
com/google/crypto/tink/streamingaead/AesCtrHmacStreamingKeyManager	&	aes128CtrHmacSha2564KBTemplate	\\
com/google/crypto/tink/streamingaead/AesCtrHmacStreamingKeyManager	&	aes256CtrHmacSha2561MBTemplate	\\
com/google/crypto/tink/streamingaead/AesCtrHmacStreamingKeyManager	&	aes256CtrHmacSha2564KBTemplate	\\
com/google/crypto/tink/streamingaead/AesCtrHmacStreamingKeyManager	&	getKeyType	\\
com/google/crypto/tink/streamingaead/AesCtrHmacStreamingKeyManager	&	getVersion	\\
com/google/crypto/tink/streamingaead/AesCtrHmacStreamingKeyManager	&	keyFactory	\\
com/google/crypto/tink/streamingaead/AesCtrHmacStreamingKeyManager	&	keyMaterialType	\\
com/google/crypto/tink/streamingaead/AesCtrHmacStreamingKeyManager	&	parseKey	\\
com/google/crypto/tink/streamingaead/AesCtrHmacStreamingKeyManager	&	register	\\
com/google/crypto/tink/streamingaead/AesCtrHmacStreamingKeyManager	&	validateKey	\\
com/google/crypto/tink/subtle/AesCtrJceCipher	&	decrypt	\\
com/google/crypto/tink/subtle/AesCtrJceCipher	&	encrypt	\\
com/google/crypto/tink/aead/AesCtrKeyManager	&	getKeyType	\\
com/google/crypto/tink/aead/AesCtrKeyManager	&	getVersion	\\
com/google/crypto/tink/aead/AesCtrKeyManager	&	keyFactory	\\
com/google/crypto/tink/aead/AesCtrKeyManager	&	keyMaterialType	\\
com/google/crypto/tink/aead/AesCtrKeyManager	&	parseKey	\\
com/google/crypto/tink/aead/AesCtrKeyManager	&	register	\\
com/google/crypto/tink/aead/AesCtrKeyManager	&	validateKey	\\
com/google/crypto/tink/subtle/AesEaxJce	&	decrypt	\\
com/google/crypto/tink/subtle/AesEaxJce	&	encrypt	\\
com/google/crypto/tink/aead/AesEaxKeyManager	&	aes128EaxTemplate	\\
com/google/crypto/tink/aead/AesEaxKeyManager	&	aes256EaxTemplate	\\
com/google/crypto/tink/aead/AesEaxKeyManager	&	getKeyType	\\
com/google/crypto/tink/aead/AesEaxKeyManager	&	getVersion	\\
com/google/crypto/tink/aead/AesEaxKeyManager	&	keyFactory	\\
com/google/crypto/tink/aead/AesEaxKeyManager	&	keyMaterialType	\\
com/google/crypto/tink/aead/AesEaxKeyManager	&	parseKey	\\
com/google/crypto/tink/aead/AesEaxKeyManager	&	rawAes128EaxTemplate	\\
com/google/crypto/tink/aead/AesEaxKeyManager	&	rawAes256EaxTemplate	\\
com/google/crypto/tink/aead/AesEaxKeyManager	&	register	\\
com/google/crypto/tink/aead/AesEaxKeyManager	&	validateKey	\\
com/google/crypto/tink/aead/subtle/AesGcmFactory	&	createAead	\\
com/google/crypto/tink/aead/subtle/AesGcmFactory	&	getKeySizeInBytes	\\
com/google/crypto/tink/subtle/AesGcmHkdfStreaming	&	expectedCiphertextSize	\\
com/google/crypto/tink/subtle/AesGcmHkdfStreaming	&	getCiphertextOffset	\\
com/google/crypto/tink/subtle/AesGcmHkdfStreaming	&	getCiphertextOverhead	\\
com/google/crypto/tink/subtle/AesGcmHkdfStreaming	&	getCiphertextSegmentSize	\\
com/google/crypto/tink/subtle/AesGcmHkdfStreaming	&	getFirstSegmentOffset	\\
com/google/crypto/tink/subtle/AesGcmHkdfStreaming	&	getHeaderLength	\\
com/google/crypto/tink/subtle/AesGcmHkdfStreaming	&	getPlaintextSegmentSize	\\
com/google/crypto/tink/subtle/AesGcmHkdfStreaming	&	newDecryptingChannel	\\
com/google/crypto/tink/subtle/AesGcmHkdfStreaming	&	newDecryptingStream	\\
com/google/crypto/tink/subtle/AesGcmHkdfStreaming	&	newEncryptingChannel	\\
com/google/crypto/tink/subtle/AesGcmHkdfStreaming	&	newEncryptingStream	\\
com/google/crypto/tink/subtle/AesGcmHkdfStreaming	&	newSeekableDecryptingChannel	\\
com/google/crypto/tink/subtle/AesGcmHkdfStreaming	&	newStreamSegmentDecrypter	\\
com/google/crypto/tink/subtle/AesGcmHkdfStreaming	&	newStreamSegmentEncrypter	\\
com/google/crypto/tink/streamingaead/AesGcmHkdfStreamingKeyManager	&	aes128GcmHkdf1MBTemplate	\\
com/google/crypto/tink/streamingaead/AesGcmHkdfStreamingKeyManager	&	aes128GcmHkdf4KBTemplate	\\
com/google/crypto/tink/streamingaead/AesGcmHkdfStreamingKeyManager	&	aes256GcmHkdf1MBTemplate	\\
com/google/crypto/tink/streamingaead/AesGcmHkdfStreamingKeyManager	&	aes256GcmHkdf4KBTemplate	\\
com/google/crypto/tink/streamingaead/AesGcmHkdfStreamingKeyManager	&	getKeyType	\\
com/google/crypto/tink/streamingaead/AesGcmHkdfStreamingKeyManager	&	getVersion	\\
com/google/crypto/tink/streamingaead/AesGcmHkdfStreamingKeyManager	&	keyFactory	\\
com/google/crypto/tink/streamingaead/AesGcmHkdfStreamingKeyManager	&	keyMaterialType	\\
com/google/crypto/tink/streamingaead/AesGcmHkdfStreamingKeyManager	&	parseKey	\\
com/google/crypto/tink/streamingaead/AesGcmHkdfStreamingKeyManager	&	register	\\
com/google/crypto/tink/streamingaead/AesGcmHkdfStreamingKeyManager	&	validateKey	\\
com/google/crypto/tink/subtle/AesGcmJce	&	decrypt	\\
com/google/crypto/tink/subtle/AesGcmJce	&	encrypt	\\
com/google/crypto/tink/aead/AesGcmKeyManager	&	aes128GcmTemplate	\\
com/google/crypto/tink/aead/AesGcmKeyManager	&	aes256GcmTemplate	\\
com/google/crypto/tink/aead/AesGcmKeyManager	&	getKeyType	\\
com/google/crypto/tink/aead/AesGcmKeyManager	&	getVersion	\\
com/google/crypto/tink/aead/AesGcmKeyManager	&	keyFactory	\\
com/google/crypto/tink/aead/AesGcmKeyManager	&	keyMaterialType	\\
com/google/crypto/tink/aead/AesGcmKeyManager	&	parseKey	\\
com/google/crypto/tink/aead/AesGcmKeyManager	&	rawAes128GcmTemplate	\\
com/google/crypto/tink/aead/AesGcmKeyManager	&	rawAes256GcmTemplate	\\
com/google/crypto/tink/aead/AesGcmKeyManager	&	register	\\
com/google/crypto/tink/aead/AesGcmKeyManager	&	validateKey	\\
com/google/crypto/tink/aead/subtle/AesGcmSiv	&	decrypt	\\
com/google/crypto/tink/aead/subtle/AesGcmSiv	&	encrypt	\\
com/google/crypto/tink/aead/AesGcmSivKeyManager	&	aes128GcmSivTemplate	\\
com/google/crypto/tink/aead/AesGcmSivKeyManager	&	aes256GcmSivTemplate	\\
com/google/crypto/tink/aead/AesGcmSivKeyManager	&	getKeyType	\\
com/google/crypto/tink/aead/AesGcmSivKeyManager	&	getVersion	\\
com/google/crypto/tink/aead/AesGcmSivKeyManager	&	keyFactory	\\
com/google/crypto/tink/aead/AesGcmSivKeyManager	&	keyMaterialType	\\
com/google/crypto/tink/aead/AesGcmSivKeyManager	&	parseKey	\\
com/google/crypto/tink/aead/AesGcmSivKeyManager	&	rawAes128GcmSivTemplate	\\
com/google/crypto/tink/aead/AesGcmSivKeyManager	&	rawAes256GcmSivTemplate	\\
com/google/crypto/tink/aead/AesGcmSivKeyManager	&	register	\\
com/google/crypto/tink/aead/AesGcmSivKeyManager	&	validateKey	\\
com/google/crypto/tink/subtle/AesSiv	&	decryptDeterministically	\\
com/google/crypto/tink/subtle/AesSiv	&	encryptDeterministically	\\
com/google/crypto/tink/daead/AesSivKeyManager	&	aes256SivTemplate	\\
com/google/crypto/tink/daead/AesSivKeyManager	&	getKeyType	\\
com/google/crypto/tink/daead/AesSivKeyManager	&	getVersion	\\
com/google/crypto/tink/daead/AesSivKeyManager	&	keyFactory	\\
com/google/crypto/tink/daead/AesSivKeyManager	&	keyMaterialType	\\
com/google/crypto/tink/daead/AesSivKeyManager	&	parseKey	\\
com/google/crypto/tink/daead/AesSivKeyManager	&	rawAes256SivTemplate	\\
com/google/crypto/tink/daead/AesSivKeyManager	&	register	\\
com/google/crypto/tink/daead/AesSivKeyManager	&	validateKey	\\
com/google/crypto/tink/integration/android/AndroidKeysetManager.Builder	&	build	\\
com/google/crypto/tink/integration/android/AndroidKeysetManager.Builder	&	doNotUseKeystore	\\
com/google/crypto/tink/integration/android/AndroidKeysetManager.Builder	&	withKeyTemplate	\\
com/google/crypto/tink/integration/android/AndroidKeysetManager.Builder	&	withKeyTemplate	\\
com/google/crypto/tink/integration/android/AndroidKeysetManager.Builder	&	withMasterKeyUri	\\
com/google/crypto/tink/integration/android/AndroidKeysetManager.Builder	&	withSharedPref	\\
com/google/crypto/tink/integration/android/AndroidKeysetManager	&	add	\\
com/google/crypto/tink/integration/android/AndroidKeysetManager	&	add	\\
com/google/crypto/tink/integration/android/AndroidKeysetManager	&	delete	\\
com/google/crypto/tink/integration/android/AndroidKeysetManager	&	destroy	\\
com/google/crypto/tink/integration/android/AndroidKeysetManager	&	disable	\\
com/google/crypto/tink/integration/android/AndroidKeysetManager	&	enable	\\
com/google/crypto/tink/integration/android/AndroidKeysetManager	&	getKeysetHandle	\\
com/google/crypto/tink/integration/android/AndroidKeysetManager	&	isUsingKeystore	\\
com/google/crypto/tink/integration/android/AndroidKeysetManager	&	promote	\\
com/google/crypto/tink/integration/android/AndroidKeysetManager	&	rotate	\\
com/google/crypto/tink/integration/android/AndroidKeysetManager	&	setPrimary	\\
com/google/crypto/tink/integration/android/AndroidKeystoreAesGcm	&	decrypt	\\
com/google/crypto/tink/integration/android/AndroidKeystoreAesGcm	&	encrypt	\\
com/google/crypto/tink/integration/android/AndroidKeystoreKmsClient.Builder	&	build	\\
com/google/crypto/tink/integration/android/AndroidKeystoreKmsClient.Builder	&	setKeyStore	\\
com/google/crypto/tink/integration/android/AndroidKeystoreKmsClient.Builder	&	setKeyUri	\\
com/google/crypto/tink/integration/android/AndroidKeystoreKmsClient	&	deleteKey	\\
com/google/crypto/tink/integration/android/AndroidKeystoreKmsClient	&	doesSupport	\\
com/google/crypto/tink/integration/android/AndroidKeystoreKmsClient	&	generateNewAeadKey	\\
com/google/crypto/tink/integration/android/AndroidKeystoreKmsClient	&	getAead	\\
com/google/crypto/tink/integration/android/AndroidKeystoreKmsClient	&	getOrGenerateNewAeadKey	\\
com/google/crypto/tink/integration/android/AndroidKeystoreKmsClient	&	withCredentials	\\
com/google/crypto/tink/integration/android/AndroidKeystoreKmsClient	&	withDefaultCredentials	\\
com/google/crypto/tink/subtle/Base64	&	decode	\\
com/google/crypto/tink/subtle/Base64	&	decode	\\
com/google/crypto/tink/subtle/Base64	&	decode	\\
com/google/crypto/tink/subtle/Base64	&	decode	\\
com/google/crypto/tink/subtle/Base64	&	encode	\\
com/google/crypto/tink/subtle/Base64	&	encode	\\
com/google/crypto/tink/subtle/Base64	&	encode	\\
com/google/crypto/tink/subtle/Base64	&	encodeToString	\\
com/google/crypto/tink/subtle/Base64	&	encodeToString	\\
com/google/crypto/tink/subtle/Base64	&	urlSafeDecode	\\
com/google/crypto/tink/subtle/Base64	&	urlSafeEncode	\\
com/google/crypto/tink/BinaryKeysetReader	&	read	\\
com/google/crypto/tink/BinaryKeysetReader	&	readEncrypted	\\
com/google/crypto/tink/BinaryKeysetReader	&	withBytes	\\
com/google/crypto/tink/BinaryKeysetReader	&	withFile	\\
com/google/crypto/tink/BinaryKeysetReader	&	withInputStream	\\
com/google/crypto/tink/BinaryKeysetWriter	&	withFile	\\
com/google/crypto/tink/BinaryKeysetWriter	&	withOutputStream	\\
com/google/crypto/tink/BinaryKeysetWriter	&	write	\\
com/google/crypto/tink/BinaryKeysetWriter	&	write	\\
com/google/crypto/tink/subtle/Bytes	&	byteArrayToInt	\\
com/google/crypto/tink/subtle/Bytes	&	byteArrayToInt	\\
com/google/crypto/tink/subtle/Bytes	&	byteArrayToInt	\\
com/google/crypto/tink/subtle/Bytes	&	concat	\\
com/google/crypto/tink/subtle/Bytes	&	equal	\\
com/google/crypto/tink/subtle/Bytes	&	intToByteArray	\\
com/google/crypto/tink/subtle/Bytes	&	xor	\\
com/google/crypto/tink/subtle/Bytes	&	xor	\\
com/google/crypto/tink/subtle/Bytes	&	xor	\\
com/google/crypto/tink/subtle/Bytes	&	xorEnd	\\
com/google/crypto/tink/Catalogue	&	getKeyManager	\\
com/google/crypto/tink/Catalogue	&	getPrimitiveWrapper	\\
com/google/crypto/tink/subtle/ChaCha20Poly1305	&	decrypt	\\
com/google/crypto/tink/subtle/ChaCha20Poly1305	&	encrypt	\\
com/google/crypto/tink/aead/ChaCha20Poly1305KeyManager	&	chaCha20Poly1305Template	\\
com/google/crypto/tink/aead/ChaCha20Poly1305KeyManager	&	getKeyType	\\
com/google/crypto/tink/aead/ChaCha20Poly1305KeyManager	&	getVersion	\\
com/google/crypto/tink/aead/ChaCha20Poly1305KeyManager	&	keyFactory	\\
com/google/crypto/tink/aead/ChaCha20Poly1305KeyManager	&	keyMaterialType	\\
com/google/crypto/tink/aead/ChaCha20Poly1305KeyManager	&	parseKey	\\
com/google/crypto/tink/aead/ChaCha20Poly1305KeyManager	&	rawChaCha20Poly1305Template	\\
com/google/crypto/tink/aead/ChaCha20Poly1305KeyManager	&	register	\\
com/google/crypto/tink/aead/ChaCha20Poly1305KeyManager	&	validateKey	\\
com/google/crypto/tink/CleartextKeysetHandle	&	fromKeyset	\\
com/google/crypto/tink/CleartextKeysetHandle	&	getKeyset	\\
com/google/crypto/tink/CleartextKeysetHandle	&	parseFrom	\\
com/google/crypto/tink/CleartextKeysetHandle	&	read	\\
com/google/crypto/tink/CleartextKeysetHandle	&	write	\\
com/google/crypto/tink/Config	&	getTinkKeyTypeEntry	\\
com/google/crypto/tink/Config	&	register	\\
com/google/crypto/tink/Config	&	registerKeyType	\\
com/google/crypto/tink/CryptoFormat	&	getOutputPrefix	\\
com/google/crypto/tink/DeterministicAead	&	decryptDeterministically	\\
com/google/crypto/tink/DeterministicAead	&	encryptDeterministically	\\
com/google/crypto/tink/daead/DeterministicAeadConfig	&	init	\\
com/google/crypto/tink/daead/DeterministicAeadConfig	&	register	\\
com/google/crypto/tink/daead/DeterministicAeadFactory	&	getPrimitive	\\
com/google/crypto/tink/daead/DeterministicAeadFactory	&	getPrimitive	\\
com/google/crypto/tink/daead/DeterministicAeadKeyTemplates	&	createAesSivKeyTemplate	\\
com/google/crypto/tink/daead/DeterministicAeadWrapper	&	getInputPrimitiveClass	\\
com/google/crypto/tink/daead/DeterministicAeadWrapper	&	getPrimitiveClass	\\
com/google/crypto/tink/daead/DeterministicAeadWrapper	&	register	\\
com/google/crypto/tink/daead/DeterministicAeadWrapper	&	wrap	\\
com/google/crypto/tink/subtle/EcdsaSignJce	&	sign	\\
com/google/crypto/tink/signature/EcdsaSignKeyManager	&	createKeyTemplate	\\
com/google/crypto/tink/signature/EcdsaSignKeyManager	&	ecdsaP256Template	\\
com/google/crypto/tink/signature/EcdsaSignKeyManager	&	getKeyType	\\
com/google/crypto/tink/signature/EcdsaSignKeyManager	&	getPublicKey	\\
com/google/crypto/tink/signature/EcdsaSignKeyManager	&	getVersion	\\
com/google/crypto/tink/signature/EcdsaSignKeyManager	&	keyFactory	\\
com/google/crypto/tink/signature/EcdsaSignKeyManager	&	keyMaterialType	\\
com/google/crypto/tink/signature/EcdsaSignKeyManager	&	parseKey	\\
com/google/crypto/tink/signature/EcdsaSignKeyManager	&	rawEcdsaP256Template	\\
com/google/crypto/tink/signature/EcdsaSignKeyManager	&	registerPair	\\
com/google/crypto/tink/signature/EcdsaSignKeyManager	&	validateKey	\\
com/google/crypto/tink/subtle/EcdsaVerifyJce	&	verify	\\
com/google/crypto/tink/subtle/EciesAeadHkdfDemHelper	&	getAead	\\
com/google/crypto/tink/subtle/EciesAeadHkdfDemHelper	&	getSymmetricKeySizeInBytes	\\
com/google/crypto/tink/subtle/EciesAeadHkdfHybridDecrypt	&	decrypt	\\
com/google/crypto/tink/subtle/EciesAeadHkdfHybridEncrypt	&	encrypt	\\
com/google/crypto/tink/hybrid/EciesAeadHkdfPrivateKeyManager	&	eciesP256HkdfHmacSha256Aes128CtrHmacSha256Template	\\
com/google/crypto/tink/hybrid/EciesAeadHkdfPrivateKeyManager	&	eciesP256HkdfHmacSha256Aes128GcmTemplate	\\
com/google/crypto/tink/hybrid/EciesAeadHkdfPrivateKeyManager	&	getKeyType	\\
com/google/crypto/tink/hybrid/EciesAeadHkdfPrivateKeyManager	&	getPublicKey	\\
com/google/crypto/tink/hybrid/EciesAeadHkdfPrivateKeyManager	&	getVersion	\\
com/google/crypto/tink/hybrid/EciesAeadHkdfPrivateKeyManager	&	keyFactory	\\
com/google/crypto/tink/hybrid/EciesAeadHkdfPrivateKeyManager	&	keyMaterialType	\\
com/google/crypto/tink/hybrid/EciesAeadHkdfPrivateKeyManager	&	parseKey	\\
com/google/crypto/tink/hybrid/EciesAeadHkdfPrivateKeyManager	&	rawEciesP256HkdfHmacSha256Aes128CtrHmacSha256CompressedTemplate	\\
com/google/crypto/tink/hybrid/EciesAeadHkdfPrivateKeyManager	&	rawEciesP256HkdfHmacSha256Aes128GcmCompressedTemplate	\\
com/google/crypto/tink/hybrid/EciesAeadHkdfPrivateKeyManager	&	registerPair	\\
com/google/crypto/tink/hybrid/EciesAeadHkdfPrivateKeyManager	&	validateKey	\\
com/google/crypto/tink/subtle/EciesHkdfRecipientKem	&	generateKey	\\
com/google/crypto/tink/subtle/EciesHkdfSenderKem.KemKey	&	getKemBytes	\\
com/google/crypto/tink/subtle/EciesHkdfSenderKem.KemKey	&	getSymmetricKey	\\
com/google/crypto/tink/subtle/EciesHkdfSenderKem	&	generateKey	\\
com/google/crypto/tink/signature/Ed25519PrivateKeyManager	&	ed25519Template	\\
com/google/crypto/tink/signature/Ed25519PrivateKeyManager	&	getKeyType	\\
com/google/crypto/tink/signature/Ed25519PrivateKeyManager	&	getPublicKey	\\
com/google/crypto/tink/signature/Ed25519PrivateKeyManager	&	getVersion	\\
com/google/crypto/tink/signature/Ed25519PrivateKeyManager	&	keyFactory	\\
com/google/crypto/tink/signature/Ed25519PrivateKeyManager	&	keyMaterialType	\\
com/google/crypto/tink/signature/Ed25519PrivateKeyManager	&	parseKey	\\
com/google/crypto/tink/signature/Ed25519PrivateKeyManager	&	rawEd25519Template	\\
com/google/crypto/tink/signature/Ed25519PrivateKeyManager	&	registerPair	\\
com/google/crypto/tink/signature/Ed25519PrivateKeyManager	&	validateKey	\\
com/google/crypto/tink/subtle/Ed25519Sign.KeyPair	&	getPrivateKey	\\
com/google/crypto/tink/subtle/Ed25519Sign.KeyPair	&	getPublicKey	\\
com/google/crypto/tink/subtle/Ed25519Sign.KeyPair	&	newKeyPair	\\
com/google/crypto/tink/subtle/Ed25519Sign	&	sign	\\
com/google/crypto/tink/subtle/Ed25519Verify	&	verify	\\
com/google/crypto/tink/subtle/EllipticCurves.CurveType	&	valueOf	\\
com/google/crypto/tink/subtle/EllipticCurves.CurveType	&	values	\\
com/google/crypto/tink/subtle/EllipticCurves.EcdsaEncoding	&	valueOf	\\
com/google/crypto/tink/subtle/EllipticCurves.EcdsaEncoding	&	values	\\
com/google/crypto/tink/subtle/EllipticCurves.PointFormatType	&	valueOf	\\
com/google/crypto/tink/subtle/EllipticCurves.PointFormatType	&	values	\\
com/google/crypto/tink/subtle/EllipticCurves	&	computeSharedSecret	\\
com/google/crypto/tink/subtle/EllipticCurves	&	computeSharedSecret	\\
com/google/crypto/tink/subtle/EllipticCurves	&	ecdsaDer2Ieee	\\
com/google/crypto/tink/subtle/EllipticCurves	&	ecdsaIeee2Der	\\
com/google/crypto/tink/subtle/EllipticCurves	&	ecPointDecode	\\
com/google/crypto/tink/subtle/EllipticCurves	&	encodingSizeInBytes	\\
com/google/crypto/tink/subtle/EllipticCurves	&	fieldSizeInBytes	\\
com/google/crypto/tink/subtle/EllipticCurves	&	generateKeyPair	\\
com/google/crypto/tink/subtle/EllipticCurves	&	generateKeyPair	\\
com/google/crypto/tink/subtle/EllipticCurves	&	getCurveSpec	\\
com/google/crypto/tink/subtle/EllipticCurves	&	getEcPrivateKey	\\
com/google/crypto/tink/subtle/EllipticCurves	&	getEcPrivateKey	\\
com/google/crypto/tink/subtle/EllipticCurves	&	getEcPublicKey	\\
com/google/crypto/tink/subtle/EllipticCurves	&	getEcPublicKey	\\
com/google/crypto/tink/subtle/EllipticCurves	&	getEcPublicKey	\\
com/google/crypto/tink/subtle/EllipticCurves	&	getEcPublicKey	\\
com/google/crypto/tink/subtle/EllipticCurves	&	getModulus	\\
com/google/crypto/tink/subtle/EllipticCurves	&	getNistP256Params	\\
com/google/crypto/tink/subtle/EllipticCurves	&	getNistP384Params	\\
com/google/crypto/tink/subtle/EllipticCurves	&	getNistP521Params	\\
com/google/crypto/tink/subtle/EllipticCurves	&	getY	\\
com/google/crypto/tink/subtle/EllipticCurves	&	isNistEcParameterSpec	\\
com/google/crypto/tink/subtle/EllipticCurves	&	isSameEcParameterSpec	\\
com/google/crypto/tink/subtle/EllipticCurves	&	isValidDerEncoding	\\
com/google/crypto/tink/subtle/EllipticCurves	&	modSqrt	\\
com/google/crypto/tink/subtle/EllipticCurves	&	pointDecode	\\
com/google/crypto/tink/subtle/EllipticCurves	&	pointDecode	\\
com/google/crypto/tink/subtle/EllipticCurves	&	pointEncode	\\
com/google/crypto/tink/subtle/EllipticCurves	&	pointEncode	\\
com/google/crypto/tink/subtle/EllipticCurves	&	validatePublicKey	\\
com/google/crypto/tink/subtle/EncryptThenAuthenticate	&	decrypt	\\
com/google/crypto/tink/subtle/EncryptThenAuthenticate	&	encrypt	\\
com/google/crypto/tink/subtle/EncryptThenAuthenticate	&	newAesCtrHmac	\\
com/google/crypto/tink/subtle/EngineFactory	&	getCustomCipherProvider	\\
com/google/crypto/tink/subtle/EngineFactory	&	getCustomKeyAgreementProvider	\\
com/google/crypto/tink/subtle/EngineFactory	&	getCustomKeyFactoryProvider	\\
com/google/crypto/tink/subtle/EngineFactory	&	getCustomKeyPairGeneratorProvider	\\
com/google/crypto/tink/subtle/EngineFactory	&	getCustomMacProvider	\\
com/google/crypto/tink/subtle/EngineFactory	&	getCustomMessageDigestProvider	\\
com/google/crypto/tink/subtle/EngineFactory	&	getCustomSignatureProvider	\\
com/google/crypto/tink/subtle/EngineFactory	&	getInstance	\\
com/google/crypto/tink/subtle/EngineFactory	&	toProviderList	\\
com/google/crypto/tink/subtle/EngineWrapper.TCipher	&	getInstance	\\
com/google/crypto/tink/subtle/EngineWrapper.TKeyAgreement	&	getInstance	\\
com/google/crypto/tink/subtle/EngineWrapper.TKeyFactory	&	getInstance	\\
com/google/crypto/tink/subtle/EngineWrapper.TKeyPairGenerator	&	getInstance	\\
com/google/crypto/tink/subtle/EngineWrapper.TMac	&	getInstance	\\
com/google/crypto/tink/subtle/EngineWrapper.TMessageDigest	&	getInstance	\\
com/google/crypto/tink/subtle/EngineWrapper.TSignature	&	getInstance	\\
com/google/crypto/tink/subtle/EngineWrapper	&	getInstance	\\
com/google/crypto/tink/subtle/Enums.HashType	&	valueOf	\\
com/google/crypto/tink/subtle/Enums.HashType	&	values	\\
com/google/crypto/tink/subtle/Hex	&	decode	\\
com/google/crypto/tink/subtle/Hex	&	encode	\\
com/google/crypto/tink/subtle/Hkdf	&	computeEciesHkdfSymmetricKey	\\
com/google/crypto/tink/subtle/Hkdf	&	computeHkdf	\\
com/google/crypto/tink/prf/HkdfPrfKeyManager	&	getKeyType	\\
com/google/crypto/tink/prf/HkdfPrfKeyManager	&	getVersion	\\
com/google/crypto/tink/prf/HkdfPrfKeyManager	&	hkdfSha256Template	\\
com/google/crypto/tink/prf/HkdfPrfKeyManager	&	keyFactory	\\
com/google/crypto/tink/prf/HkdfPrfKeyManager	&	keyMaterialType	\\
com/google/crypto/tink/prf/HkdfPrfKeyManager	&	parseKey	\\
com/google/crypto/tink/prf/HkdfPrfKeyManager	&	register	\\
com/google/crypto/tink/prf/HkdfPrfKeyManager	&	staticKeyType	\\
com/google/crypto/tink/prf/HkdfPrfKeyManager	&	validateKey	\\
com/google/crypto/tink/subtle/prf/HkdfStreamingPrf	&	computePrf	\\
com/google/crypto/tink/mac/HmacKeyManager	&	getKeyType	\\
com/google/crypto/tink/mac/HmacKeyManager	&	getVersion	\\
com/google/crypto/tink/mac/HmacKeyManager	&	hmacSha256HalfDigestTemplate	\\
com/google/crypto/tink/mac/HmacKeyManager	&	hmacSha256Template	\\
com/google/crypto/tink/mac/HmacKeyManager	&	hmacSha512HalfDigestTemplate	\\
com/google/crypto/tink/mac/HmacKeyManager	&	hmacSha512Template	\\
com/google/crypto/tink/mac/HmacKeyManager	&	keyFactory	\\
com/google/crypto/tink/mac/HmacKeyManager	&	keyMaterialType	\\
com/google/crypto/tink/mac/HmacKeyManager	&	parseKey	\\
com/google/crypto/tink/mac/HmacKeyManager	&	register	\\
com/google/crypto/tink/mac/HmacKeyManager	&	validateKey	\\
com/google/crypto/tink/prf/HmacPrfKeyManager	&	getKeyType	\\
com/google/crypto/tink/prf/HmacPrfKeyManager	&	getVersion	\\
com/google/crypto/tink/prf/HmacPrfKeyManager	&	hmacSha256Template	\\
com/google/crypto/tink/prf/HmacPrfKeyManager	&	hmacSha512Template	\\
com/google/crypto/tink/prf/HmacPrfKeyManager	&	keyFactory	\\
com/google/crypto/tink/prf/HmacPrfKeyManager	&	keyMaterialType	\\
com/google/crypto/tink/prf/HmacPrfKeyManager	&	parseKey	\\
com/google/crypto/tink/prf/HmacPrfKeyManager	&	register	\\
com/google/crypto/tink/prf/HmacPrfKeyManager	&	validateKey	\\
com/google/crypto/tink/hybrid/HybridConfig	&	init	\\
com/google/crypto/tink/hybrid/HybridConfig	&	register	\\
com/google/crypto/tink/HybridDecrypt	&	decrypt	\\
com/google/crypto/tink/hybrid/HybridDecryptFactory	&	getPrimitive	\\
com/google/crypto/tink/hybrid/HybridDecryptFactory	&	getPrimitive	\\
com/google/crypto/tink/hybrid/HybridDecryptWrapper	&	getInputPrimitiveClass	\\
com/google/crypto/tink/hybrid/HybridDecryptWrapper	&	getPrimitiveClass	\\
com/google/crypto/tink/hybrid/HybridDecryptWrapper	&	register	\\
com/google/crypto/tink/hybrid/HybridDecryptWrapper	&	wrap	\\
com/google/crypto/tink/HybridEncrypt	&	encrypt	\\
com/google/crypto/tink/hybrid/HybridEncryptFactory	&	getPrimitive	\\
com/google/crypto/tink/hybrid/HybridEncryptFactory	&	getPrimitive	\\
com/google/crypto/tink/hybrid/HybridKeyTemplates	&	createEciesAeadHkdfKeyTemplate	\\
com/google/crypto/tink/hybrid/HybridKeyTemplates	&	createEciesAeadHkdfParams	\\
com/google/crypto/tink/subtle/ImmutableByteArray	&	getBytes	\\
com/google/crypto/tink/subtle/ImmutableByteArray	&	getLength	\\
com/google/crypto/tink/subtle/ImmutableByteArray	&	of	\\
com/google/crypto/tink/subtle/ImmutableByteArray	&	of	\\
com/google/crypto/tink/subtle/IndCpaCipher	&	decrypt	\\
com/google/crypto/tink/subtle/IndCpaCipher	&	encrypt	\\
com/google/crypto/tink/JsonKeysetReader	&	read	\\
com/google/crypto/tink/JsonKeysetReader	&	readEncrypted	\\
com/google/crypto/tink/JsonKeysetReader	&	withBytes	\\
com/google/crypto/tink/JsonKeysetReader	&	withFile	\\
com/google/crypto/tink/JsonKeysetReader	&	withInputStream	\\
com/google/crypto/tink/JsonKeysetReader	&	withJsonObject	\\
com/google/crypto/tink/JsonKeysetReader	&	withPath	\\
com/google/crypto/tink/JsonKeysetReader	&	withPath	\\
com/google/crypto/tink/JsonKeysetReader	&	withString	\\
com/google/crypto/tink/JsonKeysetReader	&	withUrlSafeBase64	\\
com/google/crypto/tink/JsonKeysetWriter	&	withFile	\\
com/google/crypto/tink/JsonKeysetWriter	&	withOutputStream	\\
com/google/crypto/tink/JsonKeysetWriter	&	withPath	\\
com/google/crypto/tink/JsonKeysetWriter	&	withPath	\\
com/google/crypto/tink/JsonKeysetWriter	&	write	\\
com/google/crypto/tink/JsonKeysetWriter	&	write	\\
com/google/crypto/tink/KeyManager	&	doesSupport	\\
com/google/crypto/tink/KeyManager	&	getKeyType	\\
com/google/crypto/tink/KeyManager	&	getPrimitive	\\
com/google/crypto/tink/KeyManager	&	getPrimitive	\\
com/google/crypto/tink/KeyManager	&	getPrimitiveClass	\\
com/google/crypto/tink/KeyManager	&	getVersion	\\
com/google/crypto/tink/KeyManager	&	newKey	\\
com/google/crypto/tink/KeyManager	&	newKey	\\
com/google/crypto/tink/KeyManager	&	newKeyData	\\
com/google/crypto/tink/KeyManagerImpl	&	doesSupport	\\
com/google/crypto/tink/KeyManagerImpl	&	getKeyType	\\
com/google/crypto/tink/KeyManagerImpl	&	getPrimitive	\\
com/google/crypto/tink/KeyManagerImpl	&	getPrimitive	\\
com/google/crypto/tink/KeyManagerImpl	&	getPrimitiveClass	\\
com/google/crypto/tink/KeyManagerImpl	&	getVersion	\\
com/google/crypto/tink/KeyManagerImpl	&	newKey	\\
com/google/crypto/tink/KeyManagerImpl	&	newKey	\\
com/google/crypto/tink/KeyManagerImpl	&	newKeyData	\\
com/google/crypto/tink/KeyTemplate.OutputPrefixType	&	valueOf	\\
com/google/crypto/tink/KeyTemplate.OutputPrefixType	&	values	\\
com/google/crypto/tink/KeyTemplate	&	create	\\
com/google/crypto/tink/KeyTemplate	&	getOutputPrefixType	\\
com/google/crypto/tink/KeyTemplate	&	getTypeUrl	\\
com/google/crypto/tink/KeyTemplate	&	getValue	\\
com/google/crypto/tink/KeyTypeManager.KeyFactory	&	createKey	\\
com/google/crypto/tink/KeyTypeManager.KeyFactory	&	deriveKey	\\
com/google/crypto/tink/KeyTypeManager.KeyFactory	&	getKeyFormatClass	\\
com/google/crypto/tink/KeyTypeManager.KeyFactory	&	parseKeyFormat	\\
com/google/crypto/tink/KeyTypeManager.KeyFactory	&	validateKeyFormat	\\
com/google/crypto/tink/KeyTypeManager.PrimitiveFactory	&	getPrimitive	\\
com/google/crypto/tink/KeyTypeManager	&	getKeyClass	\\
com/google/crypto/tink/KeyTypeManager	&	getKeyType	\\
com/google/crypto/tink/KeyTypeManager	&	getPrimitive	\\
com/google/crypto/tink/KeyTypeManager	&	getVersion	\\
com/google/crypto/tink/KeyTypeManager	&	keyFactory	\\
com/google/crypto/tink/KeyTypeManager	&	keyMaterialType	\\
com/google/crypto/tink/KeyTypeManager	&	parseKey	\\
com/google/crypto/tink/KeyTypeManager	&	supportedPrimitives	\\
com/google/crypto/tink/KeyTypeManager	&	validateKey	\\
com/google/crypto/tink/KeysetHandle	&	assertEnoughEncryptedKeyMaterial	\\
com/google/crypto/tink/KeysetHandle	&	assertEnoughKeyMaterial	\\
com/google/crypto/tink/KeysetHandle	&	generateNew	\\
com/google/crypto/tink/KeysetHandle	&	generateNew	\\
com/google/crypto/tink/KeysetHandle	&	getKeysetInfo	\\
com/google/crypto/tink/KeysetHandle	&	getPrimitive	\\
com/google/crypto/tink/KeysetHandle	&	getPrimitive	\\
com/google/crypto/tink/KeysetHandle	&	getPublicKeysetHandle	\\
com/google/crypto/tink/KeysetHandle	&	read	\\
com/google/crypto/tink/KeysetHandle	&	readNoSecret	\\
com/google/crypto/tink/KeysetHandle	&	readNoSecret	\\
com/google/crypto/tink/KeysetHandle	&	toString	\\
com/google/crypto/tink/KeysetHandle	&	write	\\
com/google/crypto/tink/KeysetHandle	&	writeNoSecret	\\
com/google/crypto/tink/KeysetManager	&	add	\\
com/google/crypto/tink/KeysetManager	&	add	\\
com/google/crypto/tink/KeysetManager	&	addNewKey	\\
com/google/crypto/tink/KeysetManager	&	delete	\\
com/google/crypto/tink/KeysetManager	&	destroy	\\
com/google/crypto/tink/KeysetManager	&	disable	\\
com/google/crypto/tink/KeysetManager	&	enable	\\
com/google/crypto/tink/KeysetManager	&	getKeysetHandle	\\
com/google/crypto/tink/KeysetManager	&	promote	\\
com/google/crypto/tink/KeysetManager	&	rotate	\\
com/google/crypto/tink/KeysetManager	&	setPrimary	\\
com/google/crypto/tink/KeysetManager	&	withEmptyKeyset	\\
com/google/crypto/tink/KeysetManager	&	withKeysetHandle	\\
com/google/crypto/tink/KeysetReader	&	read	\\
com/google/crypto/tink/KeysetReader	&	readEncrypted	\\
com/google/crypto/tink/KeysetWriter	&	write	\\
com/google/crypto/tink/KeysetWriter	&	write	\\
com/google/crypto/tink/aead/KmsAeadKeyManager	&	getKeyType	\\
com/google/crypto/tink/aead/KmsAeadKeyManager	&	getVersion	\\
com/google/crypto/tink/aead/KmsAeadKeyManager	&	keyFactory	\\
com/google/crypto/tink/aead/KmsAeadKeyManager	&	keyMaterialType	\\
com/google/crypto/tink/aead/KmsAeadKeyManager	&	parseKey	\\
com/google/crypto/tink/aead/KmsAeadKeyManager	&	register	\\
com/google/crypto/tink/aead/KmsAeadKeyManager	&	validateKey	\\
com/google/crypto/tink/KmsClient	&	doesSupport	\\
com/google/crypto/tink/KmsClient	&	getAead	\\
com/google/crypto/tink/KmsClient	&	withCredentials	\\
com/google/crypto/tink/KmsClient	&	withDefaultCredentials	\\
com/google/crypto/tink/KmsClients	&	add	\\
com/google/crypto/tink/KmsClients	&	get	\\
com/google/crypto/tink/KmsClients	&	getAutoLoaded	\\
com/google/crypto/tink/aead/KmsEnvelopeAead	&	decrypt	\\
com/google/crypto/tink/aead/KmsEnvelopeAead	&	encrypt	\\
com/google/crypto/tink/aead/KmsEnvelopeAeadKeyManager	&	getKeyType	\\
com/google/crypto/tink/aead/KmsEnvelopeAeadKeyManager	&	getVersion	\\
com/google/crypto/tink/aead/KmsEnvelopeAeadKeyManager	&	keyFactory	\\
com/google/crypto/tink/aead/KmsEnvelopeAeadKeyManager	&	keyMaterialType	\\
com/google/crypto/tink/aead/KmsEnvelopeAeadKeyManager	&	parseKey	\\
com/google/crypto/tink/aead/KmsEnvelopeAeadKeyManager	&	register	\\
com/google/crypto/tink/aead/KmsEnvelopeAeadKeyManager	&	validateKey	\\
com/google/crypto/tink/subtle/Kwp	&	unwrap	\\
com/google/crypto/tink/subtle/Kwp	&	wrap	\\
com/google/crypto/tink/Mac	&	computeMac	\\
com/google/crypto/tink/Mac	&	verifyMac	\\
com/google/crypto/tink/mac/MacConfig	&	init	\\
com/google/crypto/tink/mac/MacConfig	&	register	\\
com/google/crypto/tink/mac/MacConfig	&	registerStandardKeyTypes	\\
com/google/crypto/tink/mac/MacFactory	&	getPrimitive	\\
com/google/crypto/tink/mac/MacFactory	&	getPrimitive	\\
com/google/crypto/tink/mac/MacKeyTemplates	&	createHmacKeyTemplate	\\
com/google/crypto/tink/NoSecretKeysetHandle	&	parseFrom	\\
com/google/crypto/tink/NoSecretKeysetHandle	&	read	\\
com/google/crypto/tink/subtle/PemKeyType	&	readKey	\\
com/google/crypto/tink/subtle/PemKeyType	&	valueOf	\\
com/google/crypto/tink/subtle/PemKeyType	&	values	\\
com/google/crypto/tink/prf/Prf	&	compute	\\
com/google/crypto/tink/subtle/PrfAesCmac	&	compute	\\
com/google/crypto/tink/prf/PrfConfig	&	register	\\
com/google/crypto/tink/subtle/PrfHmacJce	&	compute	\\
com/google/crypto/tink/subtle/PrfHmacJce	&	getMaxOutputLength	\\
com/google/crypto/tink/subtle/prf/PrfImpl	&	compute	\\
com/google/crypto/tink/subtle/prf/PrfImpl	&	wrap	\\
com/google/crypto/tink/subtle/PrfMac	&	computeMac	\\
com/google/crypto/tink/subtle/PrfMac	&	verifyMac	\\
com/google/crypto/tink/prf/PrfSet	&	computePrimary	\\
com/google/crypto/tink/prf/PrfSet	&	getPrfs	\\
com/google/crypto/tink/prf/PrfSet	&	getPrimaryId	\\
com/google/crypto/tink/prf/PrfSetWrapper	&	getInputPrimitiveClass	\\
com/google/crypto/tink/prf/PrfSetWrapper	&	getPrimitiveClass	\\
com/google/crypto/tink/prf/PrfSetWrapper	&	register	\\
com/google/crypto/tink/prf/PrfSetWrapper	&	wrap	\\
com/google/crypto/tink/PrimitiveSet.Entry	&	getIdentifier	\\
com/google/crypto/tink/PrimitiveSet.Entry	&	getKeyId	\\
com/google/crypto/tink/PrimitiveSet.Entry	&	getOutputPrefixType	\\
com/google/crypto/tink/PrimitiveSet.Entry	&	getPrimitive	\\
com/google/crypto/tink/PrimitiveSet.Entry	&	getStatus	\\
com/google/crypto/tink/PrimitiveWrapper	&	getInputPrimitiveClass	\\
com/google/crypto/tink/PrimitiveWrapper	&	getPrimitiveClass	\\
com/google/crypto/tink/PrimitiveWrapper	&	wrap	\\
com/google/crypto/tink/PrivateKeyManager	&	getPublicKeyData	\\
com/google/crypto/tink/PrivateKeyManagerImpl	&	getPublicKeyData	\\
com/google/crypto/tink/PrivateKeyTypeManager	&	getPublicKey	\\
com/google/crypto/tink/PrivateKeyTypeManager	&	getPublicKeyClass	\\
com/google/crypto/tink/PublicKeySign	&	sign	\\
com/google/crypto/tink/signature/PublicKeySignFactory	&	getPrimitive	\\
com/google/crypto/tink/signature/PublicKeySignFactory	&	getPrimitive	\\
com/google/crypto/tink/signature/PublicKeySignWrapper	&	getInputPrimitiveClass	\\
com/google/crypto/tink/signature/PublicKeySignWrapper	&	getPrimitiveClass	\\
com/google/crypto/tink/signature/PublicKeySignWrapper	&	register	\\
com/google/crypto/tink/signature/PublicKeySignWrapper	&	wrap	\\
com/google/crypto/tink/PublicKeyVerify	&	verify	\\
com/google/crypto/tink/signature/PublicKeyVerifyFactory	&	getPrimitive	\\
com/google/crypto/tink/signature/PublicKeyVerifyFactory	&	getPrimitive	\\
com/google/crypto/tink/subtle/Random	&	randBytes	\\
com/google/crypto/tink/subtle/Random	&	randInt	\\
com/google/crypto/tink/subtle/Random	&	randInt	\\
com/google/crypto/tink/Registry	&	addCatalogue	\\
com/google/crypto/tink/Registry	&	getCatalogue	\\
com/google/crypto/tink/Registry	&	getInputPrimitive	\\
com/google/crypto/tink/Registry	&	getKeyManager	\\
com/google/crypto/tink/Registry	&	getKeyManager	\\
com/google/crypto/tink/Registry	&	getPrimitive	\\
com/google/crypto/tink/Registry	&	getPrimitive	\\
com/google/crypto/tink/Registry	&	getPrimitive	\\
com/google/crypto/tink/Registry	&	getPrimitive	\\
com/google/crypto/tink/Registry	&	getPrimitive	\\
com/google/crypto/tink/Registry	&	getPrimitive	\\
com/google/crypto/tink/Registry	&	getPrimitive	\\
com/google/crypto/tink/Registry	&	getPrimitive	\\
com/google/crypto/tink/Registry	&	getPrimitives	\\
com/google/crypto/tink/Registry	&	getPrimitives	\\
com/google/crypto/tink/Registry	&	getPublicKeyData	\\
com/google/crypto/tink/Registry	&	getUntypedKeyManager	\\
com/google/crypto/tink/Registry	&	newKey	\\
com/google/crypto/tink/Registry	&	newKey	\\
com/google/crypto/tink/Registry	&	newKeyData	\\
com/google/crypto/tink/Registry	&	newKeyData	\\
com/google/crypto/tink/Registry	&	registerAsymmetricKeyManagers	\\
com/google/crypto/tink/Registry	&	registerKeyManager	\\
com/google/crypto/tink/Registry	&	registerKeyManager	\\
com/google/crypto/tink/Registry	&	registerKeyManager	\\
com/google/crypto/tink/Registry	&	registerKeyManager	\\
com/google/crypto/tink/Registry	&	registerKeyManager	\\
com/google/crypto/tink/Registry	&	registerPrimitiveWrapper	\\
com/google/crypto/tink/Registry	&	wrap	\\
com/google/crypto/tink/Registry	&	wrap	\\
com/google/crypto/tink/subtle/RewindableReadableByteChannel	&	close	\\
com/google/crypto/tink/subtle/RewindableReadableByteChannel	&	disableRewinding	\\
com/google/crypto/tink/subtle/RewindableReadableByteChannel	&	isOpen	\\
com/google/crypto/tink/subtle/RewindableReadableByteChannel	&	read	\\
com/google/crypto/tink/subtle/RewindableReadableByteChannel	&	rewind	\\
com/google/crypto/tink/hybrid/subtle/RsaKemHybridDecrypt	&	decrypt	\\
com/google/crypto/tink/hybrid/subtle/RsaKemHybridEncrypt	&	encrypt	\\
com/google/crypto/tink/subtle/RsaSsaPkcs1SignJce	&	sign	\\
com/google/crypto/tink/signature/RsaSsaPkcs1SignKeyManager	&	getKeyType	\\
com/google/crypto/tink/signature/RsaSsaPkcs1SignKeyManager	&	getPublicKey	\\
com/google/crypto/tink/signature/RsaSsaPkcs1SignKeyManager	&	getVersion	\\
com/google/crypto/tink/signature/RsaSsaPkcs1SignKeyManager	&	keyFactory	\\
com/google/crypto/tink/signature/RsaSsaPkcs1SignKeyManager	&	keyMaterialType	\\
com/google/crypto/tink/signature/RsaSsaPkcs1SignKeyManager	&	parseKey	\\
com/google/crypto/tink/signature/RsaSsaPkcs1SignKeyManager	&	rawRsa3072SsaPkcs1Sha256F4Template	\\
com/google/crypto/tink/signature/RsaSsaPkcs1SignKeyManager	&	rawRsa4096SsaPkcs1Sha512F4Template	\\
com/google/crypto/tink/signature/RsaSsaPkcs1SignKeyManager	&	registerPair	\\
com/google/crypto/tink/signature/RsaSsaPkcs1SignKeyManager	&	rsa3072SsaPkcs1Sha256F4Template	\\
com/google/crypto/tink/signature/RsaSsaPkcs1SignKeyManager	&	rsa4096SsaPkcs1Sha512F4Template	\\
com/google/crypto/tink/signature/RsaSsaPkcs1SignKeyManager	&	validateKey	\\
com/google/crypto/tink/subtle/RsaSsaPkcs1VerifyJce	&	verify	\\
com/google/crypto/tink/subtle/RsaSsaPssSignJce	&	sign	\\
com/google/crypto/tink/signature/RsaSsaPssSignKeyManager	&	getKeyType	\\
com/google/crypto/tink/signature/RsaSsaPssSignKeyManager	&	getPublicKey	\\
com/google/crypto/tink/signature/RsaSsaPssSignKeyManager	&	getVersion	\\
com/google/crypto/tink/signature/RsaSsaPssSignKeyManager	&	keyFactory	\\
com/google/crypto/tink/signature/RsaSsaPssSignKeyManager	&	keyMaterialType	\\
com/google/crypto/tink/signature/RsaSsaPssSignKeyManager	&	parseKey	\\
com/google/crypto/tink/signature/RsaSsaPssSignKeyManager	&	rawRsa3072PssSha256F4Template	\\
com/google/crypto/tink/signature/RsaSsaPssSignKeyManager	&	rawRsa4096PssSha512F4Template	\\
com/google/crypto/tink/signature/RsaSsaPssSignKeyManager	&	registerPair	\\
com/google/crypto/tink/signature/RsaSsaPssSignKeyManager	&	rsa3072PssSha256F4Template	\\
com/google/crypto/tink/signature/RsaSsaPssSignKeyManager	&	rsa4096PssSha512F4Template	\\
com/google/crypto/tink/signature/RsaSsaPssSignKeyManager	&	validateKey	\\
com/google/crypto/tink/subtle/RsaSsaPssVerifyJce	&	verify	\\
com/google/crypto/tink/integration/android/SharedPrefKeysetReader	&	read	\\
com/google/crypto/tink/integration/android/SharedPrefKeysetReader	&	readEncrypted	\\
com/google/crypto/tink/integration/android/SharedPrefKeysetWriter	&	write	\\
com/google/crypto/tink/integration/android/SharedPrefKeysetWriter	&	write	\\
com/google/crypto/tink/signature/SignatureConfig	&	init	\\
com/google/crypto/tink/signature/SignatureConfig	&	register	\\
com/google/crypto/tink/signature/SignatureKeyTemplates	&	createEcdsaKeyTemplate	\\
com/google/crypto/tink/signature/SignatureKeyTemplates	&	createRsaSsaPkcs1KeyTemplate	\\
com/google/crypto/tink/signature/SignatureKeyTemplates	&	createRsaSsaPssKeyTemplate	\\
com/google/crypto/tink/signature/SignaturePemKeysetReader.Builder	&	addPem	\\
com/google/crypto/tink/signature/SignaturePemKeysetReader.Builder	&	build	\\
com/google/crypto/tink/signature/SignaturePemKeysetReader	&	newBuilder	\\
com/google/crypto/tink/signature/SignaturePemKeysetReader	&	read	\\
com/google/crypto/tink/signature/SignaturePemKeysetReader	&	readEncrypted	\\
com/google/crypto/tink/subtle/StreamSegmentDecrypter	&	decryptSegment	\\
com/google/crypto/tink/subtle/StreamSegmentDecrypter	&	init	\\
com/google/crypto/tink/subtle/StreamSegmentEncrypter	&	encryptSegment	\\
com/google/crypto/tink/subtle/StreamSegmentEncrypter	&	encryptSegment	\\
com/google/crypto/tink/subtle/StreamSegmentEncrypter	&	getHeader	\\
com/google/crypto/tink/StreamingAead	&	newDecryptingChannel	\\
com/google/crypto/tink/StreamingAead	&	newDecryptingStream	\\
com/google/crypto/tink/StreamingAead	&	newEncryptingChannel	\\
com/google/crypto/tink/StreamingAead	&	newEncryptingStream	\\
com/google/crypto/tink/StreamingAead	&	newSeekableDecryptingChannel	\\
com/google/crypto/tink/streamingaead/StreamingAeadConfig	&	init	\\
com/google/crypto/tink/streamingaead/StreamingAeadConfig	&	register	\\
com/google/crypto/tink/streamingaead/StreamingAeadFactory	&	getPrimitive	\\
com/google/crypto/tink/streamingaead/StreamingAeadFactory	&	getPrimitive	\\
com/google/crypto/tink/streamingaead/StreamingAeadKeyTemplates	&	createAesCtrHmacStreamingKeyTemplate	\\
com/google/crypto/tink/streamingaead/StreamingAeadKeyTemplates	&	createAesGcmHkdfStreamingKeyTemplate	\\
com/google/crypto/tink/streamingaead/StreamingAeadWrapper	&	getInputPrimitiveClass	\\
com/google/crypto/tink/streamingaead/StreamingAeadWrapper	&	getPrimitiveClass	\\
com/google/crypto/tink/streamingaead/StreamingAeadWrapper	&	register	\\
com/google/crypto/tink/streamingaead/StreamingAeadWrapper	&	wrap	\\
com/google/crypto/tink/subtle/prf/StreamingPrf	&	computePrf	\\
com/google/crypto/tink/subtle/SubtleUtil	&	androidApiLevel	\\
com/google/crypto/tink/subtle/SubtleUtil	&	bytes2Integer	\\
com/google/crypto/tink/subtle/SubtleUtil	&	integer2Bytes	\\
com/google/crypto/tink/subtle/SubtleUtil	&	isAndroid	\\
com/google/crypto/tink/subtle/SubtleUtil	&	mgf1	\\
com/google/crypto/tink/subtle/SubtleUtil	&	putAsUnsigedInt	\\
com/google/crypto/tink/subtle/SubtleUtil	&	toDigestAlgo	\\
com/google/crypto/tink/subtle/SubtleUtil	&	toEcdsaAlgo	\\
com/google/crypto/tink/subtle/SubtleUtil	&	toRsaSsaPkcs1Algo	\\
com/google/crypto/tink/config/TinkConfig	&	init	\\
com/google/crypto/tink/config/TinkConfig	&	register	\\
com/google/crypto/tink/subtle/Validators	&	validateAesKeySize	\\
com/google/crypto/tink/subtle/Validators	&	validateCryptoKeyUri	\\
com/google/crypto/tink/subtle/Validators	&	validateExists	\\
com/google/crypto/tink/subtle/Validators	&	validateKmsKeyUriAndRemovePrefix	\\
com/google/crypto/tink/subtle/Validators	&	validateNotExists	\\
com/google/crypto/tink/subtle/Validators	&	validateRsaModulusSize	\\
com/google/crypto/tink/subtle/Validators	&	validateRsaPublicExponent	\\
com/google/crypto/tink/subtle/Validators	&	validateSignatureHash	\\
com/google/crypto/tink/subtle/Validators	&	validateTypeUrl	\\
com/google/crypto/tink/subtle/Validators	&	validateVersion	\\
com/google/crypto/tink/subtle/X25519	&	computeSharedSecret	\\
com/google/crypto/tink/subtle/X25519	&	generatePrivateKey	\\
com/google/crypto/tink/subtle/X25519	&	publicFromPrivate	\\
com/google/crypto/tink/subtle/XChaCha20Poly1305	&	decrypt	\\
com/google/crypto/tink/subtle/XChaCha20Poly1305	&	encrypt	\\
com/google/crypto/tink/aead/XChaCha20Poly1305KeyManager	&	getKeyType	\\
com/google/crypto/tink/aead/XChaCha20Poly1305KeyManager	&	getVersion	\\
com/google/crypto/tink/aead/XChaCha20Poly1305KeyManager	&	keyFactory	\\
com/google/crypto/tink/aead/XChaCha20Poly1305KeyManager	&	keyMaterialType	\\
com/google/crypto/tink/aead/XChaCha20Poly1305KeyManager	&	parseKey	\\
com/google/crypto/tink/aead/XChaCha20Poly1305KeyManager	&	rawXChaCha20Poly1305Template	\\
com/google/crypto/tink/aead/XChaCha20Poly1305KeyManager	&	register	\\
com/google/crypto/tink/aead/XChaCha20Poly1305KeyManager	&	validateKey	\\
com/google/crypto/tink/aead/XChaCha20Poly1305KeyManager	&	xChaCha20Poly1305Template	\\

\end{longtable}
\end{landscape}
\endgroup

\end{document}