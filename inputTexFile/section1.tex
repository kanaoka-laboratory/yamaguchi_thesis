\newpage
\section{はじめに} 
近年開発者向けユーザブルセキュリティ、ユーザブルプライバシーの研究分野においてソフトウェア開発者の暗号技術の利用に関する研究が活発になっている。
これまでの研究では、そういった暗号技術の利用が適切にされておらず脆弱性を生み出しているソフトウェアが多数存在することが判明している。

しかしこれらの研究はSSL/TLSや DES、AESなど特定の暗号技術、ソフトウェア開発者の開発時の誤使用など特定の状況に限った調査だけであり、実際にどの程度のソフト
ウェアでどのように暗号技術が利用されているのかなどの暗号技術全般の網羅的な調査は行われていなかった。
Androidアプリケーションにおいても同様に、SSL/TLSなどの暗号技術の利用がされているケースが多くあると考えられている。

そこで本研究ではソフトウェアの暗号技術の利用状況の網羅的な調査の一環として、ApktoolやBaksmaliといったツールによる静的解析が容易であること、世界のモバイル端末におけるOSのシェア率が高いこと\cite{share}からJavaで開発されたAndroidアプリケーションを調査対象とする。

Androidアプリケーションを静的解析し、暗号で用いられるメソッド名の抽出、API の利用傾向分析を行う。

Android アプリケーションの現状はどのような暗号技術が利用され、どの程度暗号技術が利用され
ているのか、その時のアルゴリズムはどのようなものが利用されているのかまた暗号技術の利用には何らかの傾向があるのかなどを明らかにすることを目的とする。

調査にはAndroZooより取得した307,587個のAndroidアプリケーションと、
Google社のサードパーティ製APIであるTinkから抽出した暗号・セキュリティに関するメソッド705個のリストを使用する。

調査の結果、調査対象のAndroidアプリケーション群における暗号・セキュリティに関するメソッドの利用数が判明した。

本稿の構成は以下の通りである。第2章で、本研究に関連する技術などについての解説を行い、第3章では関連研究の紹介を行う。第4章では本研究を始めるにあたっての前段階の準備と調査方法や環境についての説明をし、第5章では結果と考察を行う。第6章で残課題について説明し、第7章でまとめる。
