\newpage
\section{はじめに} 
近年開発者向けユーザブルセキュリティ、ユーザブルプライバシーの研究分野においてソフトウェア開発者の暗号技術の利用に関する研究が活発になっている。
これまでの研究では、そういった暗号技術の利用が適切にされておらず脆弱性を生み出しているアプリケーション(ソフトウェア?)が多数存在することが判明している。
Androidアプリケーション内では、SSL/TLSの暗号技術の利用がされているケースが多くあると考えられている。

そこで、Androidアプリケーションの暗号技術利用に関する全体像を知るために広くデータ分析をして、現状を明らかにするという目的のもと研究を進めていく。
暗号技術の不適切利用に限った調査はいくつか行われているが、それらはSSL/TLSやDES、AESなど特定の暗号技術に限った調査だけであり、暗号技術全般の網羅的な調査が行われていないという課題がある。

本研究では、APIの分析から行っていく。先行研究である河合による調査では、一部のAPIのみで行なわれた研究であるので、より網羅的な調査のために他のライブラリやAPKのデータ規模を拡大し、
調査の幅を広げていく。

本稿の構成は以下のとおりである。始めに第2章で、本研究に関連する技術などについての解説を行い、次の第3章では関連研究の紹介を行う。第4章、第5章では本研究を始めるにあたっての前段階の準備と調査方法や環境についての説明をし、第6章では結果と考察を行う。第7章で本研究をまとめ、残課題について説明し、最後に第8章で残課題について説明する。