\newpage
\section{調査ターゲット(準備)}
\subsection {APKの取得方法}
\subsection{smaliファイル化方法}

\subsection{APK暗号技術}
 APKに使われている暗号技術は大きく分けて3種類ある。

 \subsubsection {公式API}
公式APIとは、Androidの開発者向け公式WebサイトであるAndroid DevelopersのAPIリファレンスに記載されているAPIである。

 \subsubsection {サードパーティー製API}
サードパーティー製APIとは、サードパーティが提供するAPIのことである。
サードパーティとは、特定のハードウェア、OS、ソフトウェア、あるいはサービスなどを対象として、
それに対応する製品を販売、提供している組織や企業のことを指す。
サードパーティ製APIの例としては、Google社のtinkやFacebook社のConcealnがある。

 \subsubsection {独自実装等のAPI}
API開発者が既存のAPIを利用せずに独自に実装したAPIや、先述2つに含まれないものを独自実装等のAPIと本論文では呼ぶこととする。

\subsection{   }
河合による先行研究では、Android Developers のAPI リファレンスに記載されているAPIから、暗号・セキュリティに関
するパッケージ、クラス、メソッドを抽出しリスト化した。
このリストをもとにAPKにおいてど
れほど暗号技術が利用されているかの分析を行った。

独自実装等のAPIはドキュメントが公開されている可能性
が低いためAPIのリスト化が困難である。これは、
RSAやECC、Cryptoといった暗号、セキュリティ
に関するキーワードをAPIのリストの代わりとし検
索する必要があるためAPKの網羅的調査を行う上
で困難である。

比較して、サードパーティー製API
ではドキュメントが公開されているものもあるので
リスト化の困難性が少ない。そこで、本研究では特
にサードパーティー製APIを本研究の分析対象とする。
サードパーティ製のAPIの分析ではまずAPIのリスト化を行う必要があるが、サードパーティー製
APIは公式APIとは違いドキュメントが作成され
ていないものがある。存在しない場合はサードパー
ティ製の APIのソースコードを解析し、APIのド
キュメントを作成してからAPIのリストの作成を行う。





\subsection{サードパーティー製APIの種類}
\subsubsection{Tink}

GoogleのTinkには4つのプリミティブがある。

    関連データを備えた認証付き暗号(プリミティブ: AEAD)

    メッセージ認証コード (プリミティブ: MAC)

    ディジタル署名(プリミティブ: PublicKeySignとPublicKeyVerify)

    ハイブリッド暗号化(プリミティブ: HybridEncryptとHybridDecrypt)


\subsection{APIの取得方法}
Tinkのドキュメントページ[ ]からAPIを抽出する。そのクラス(計167個)が持つ
メソッド計000個のリスト化を行った。このリストは、2020年00月のものである。
リストの1部を抜粋し、表0に示す。リスト全体は付録Aに示す。


\if
 APIの分析
-公式APIの分析
河合さんがやった
  -サードパーティ製APIの分析
   今回は、Google社のAPI、Tinkに対して研究を進めていく
  -独自実装等のAPIの分析
   残課題
\fi