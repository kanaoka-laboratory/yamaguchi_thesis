\newpage
\section{調査対象と手法}
%%%%%%%%%%%%%%%%%%%%
\subsection {smaliファイル}

調査対象となるAPK群は、AndroZoo\cite{}のデータセットより411,486個のAPKから展開されたsmaliファイルを使用する。

%%%%%%%%%%%%%%%%%%%%
\subsection{Android APIの分類}
APKに使われているAPIは大きく分けて3種類存在する。
%%%%%%%%%%
 \subsubsection {公式API}
\if0
・Androidでは、ソフトウェア開発のために必要なプログラムやライブラリをGoogle社がAndroid SDKとして提供してる
・SDKで提供されるライブラリはAndroid開発者向けサイトAndroid DevelopersにAPIリファレンスとして記載されてる
・この研究ではここに記載されてるAPIを「公式API」と呼ぶことにするよ
\fi
Androidの開発者向け公式WebサイトであるAndroid DevelopersのAPIリファレンスに記載されているAPIのことを本研究では公式APIと呼ぶこととする。
%%%%%%%%%%
 \subsubsection {サードパーティー製API}
サードパーティー製APIとは、サードパーティが提供するAPIのことである。
サードパーティとは、特定のハードウェア、OS、ソフトウェア、あるいはサービスなどを対象として、
それに対応する製品を販売、提供している組織や企業のことを指す。
%%%%%%%%%%
 \subsubsection {独自実装等のAPI}
API開発者が既存のAPIを利用せずに独自に実装したAPIや、先述2つに含まれないものを独自実装等のAPIと本論文では呼ぶこととする。

%%%%%%%%%%%%%%%%%%%%
\subsection{APIの分析}
独自実装等のAPIはドキュメントが公開されている可能性
が低いためAPIのリスト化が困難である。これは、
RSAやECC、Cryptoといった暗号、セキュリティ
に関するキーワードをAPIのリストの代わりとし検
索する必要があるためAPKの網羅的調査を行う上
で困難である。

比較して、サードパーティー製API
ではドキュメントが公開されているものもあるので
リスト化の困難性が少ない。
サードパーティ製のAPIの分析ではまずAPIのリスト化を行う必要があるが、サードパーティー製
APIは公式APIとは違いドキュメントが作成され
ていないものがある。存在しない場合はサードパー
ティ製の APIのソースコードを解析し、APIのド
キュメントを作成してからAPIのリストの作成を行う。

独自実装等のAPIは今後の課題とする。

%%%%%%%%%%%%%%%%%%%%
\subsection{サードパーティ製APIの分析}

%%%%%%%%%%
\subsubsection{サードパーティ製APIの例}
どんなサードパーティ製APIが存在するのか紹介する。
\begin{itemize}
\item Tink 

Tinkは、Googleの暗号技術者とセキュリティエンジニアのグループが開発した、
多言語でクロスプラットフォームな暗号ライブラリである。

\item Conceal

Concealは、Facebookが開発したライブラリである。共通鍵暗号アルゴリズム AES(256bit)と暗号利用モードGCMを用いた暗号化処理を代行している。
 
\end{itemize}
この中でもTinkは、Android OSを提供しているGoogle社によるサードパーティ製APIであるため、Androidアプリケーション開発者にも利用されている可能性は高いと考えられる。本研究ではTinkを調査対象とする。

%%%%%%%%%%
\subsubsection{Tinkの分析}

Tinkは現在、それぞれのプリミティブを使って実装された、4つの暗号化操作を提供している。
\begin{itemize}
\item 関連データを備えた認証付き暗号(プリミティブ: AEAD)
\item メッセージ認証コード (プリミティブ: MAC)
\item ディジタル署名(プリミティブ: PublicKeySignとPublicKeyVerify)
\item ハイブリッド暗号化(プリミティブ: HybridEncryptとHybridDecrypt)
\end{itemize}
プリミティブとは、単純あるいは基本的な構造や要素のことを言う。
Tinkには、ドキュメントが存在するので、ドキュメントが存在しない
サードパーティ製APIよりリスト化の困難性が少ない。

\if0
\subsubsection{それぞれのプリミティブの詳しい説明いれるか迷っている}
暗号の説明をしていく。
\begin{itemize}
\item AEAD

AEADは、次の 3 つのアルゴリズムの組 AE = (AE-K, AE-E, AE-D) で定義される。

\item MAC

メッセージ認証コードMACは、Message Authentication Codeの略であり、ネットワークを通じて伝送されたメッセージが途中で改竄されていないかを確認することである。

\item PublicKeySignとPublicKeyVerify

\item HybridEncryptとHybridDecrypt

\end{itemize}
\fi
%%%%%%%%%%%%%%%%%%%%
\subsection{APIの取得方法}
Tinkのドキュメント\cite{Tink Cryptography}からAPIを抽出する。その167個のクラスが持つメソッド705個のリスト化を行った。このリストは、2020年12月のものである。
リストの1部を抜粋し、表\ref{tb:APImethod}に示す。リスト全体は付録\ref{tab:HTMLList}に示す。

\begin{table}[t]
\begin{center}
\caption{Tinkのドキュメントページより取得した暗号・セキュリティに関するクラスが持つ705個のメソッドのリストの一部抜粋}
\begin{tabular}{ll} \hline
クラス名 & メソッド名(引数) \\ \hline
AesEaxKeyManager & validateKey() \\
AndroidKeysetManager & getKeysetHandle() \\
AndroidKeystoreAesGcm &  encrypt\\ \hline
\end{tabular}
\label{tb:APImethod}
\end{center}
\end{table}



\subsection{APIリストとsmaliファイルで利用されるメソッドのマッチング調査}
APIリストとsmaliファイルで利用されるメソッドのマッチングは、
まずsmaliファイルに記述された情報から”invoke”が記述された行を以下の正規表現にて抽出する。

\begin{center} .* invoke-.*, L \end{center}
(任意の文字列と''invoke-''と任意の文字列と'' ''と''L'')

抽出された行の例を以下に示す。

\begin{center} invoke-virtual \{p0\}, Lcom/google/crypto/tink/integration/android/AndroidKeysetManager;->getKeysetHandle()Lcom/google/crypto/tink/KeysetHandle;\end{center}

抽出された行はsmaliファイルの形式で記述されているため、マッチングの際に不必要な情報や
マッチングに適さない形となっている。上記の抽出された行のをJavaの形式で表現すると以下になる。

\begin{center} じゃばああああああああああああああああ\end{center}

\lstinputlisting[caption=APIリストとsmaliファイルで利用されるメソッドのマッチンPythonスクリプト,label=find_grepbase.sh]{C:/Users/arier/Documents/卒論/yamaguchi_thesis-main/コード/grepbase.sh}




































