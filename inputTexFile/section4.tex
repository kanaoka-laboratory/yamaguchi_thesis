\newpage
\section{調査ターゲット(準備)}
\subsection {APKの取得方法}
\subsection{smaliファイル化方法}

\subsection{APK暗号技術}
 APKに使われている暗号技術は大きく分けて3種類ある。

 \subsubsection {公式API}
公式APIとは、Androidの開発者向け公式WebサイトであるAndroid DevelopersのAPIリファレンスに記載されているAPIである。

 \subsubsection {サードパーティー製API}
サードパーティー製APIとは、サードパーティが提供するAPIのことである。
サードパーティとは、特定のハードウェア、OS、ソフトウェア、あるいはサービスなどを対象として、
それに対応する製品を販売、提供している組織や企業のことを指す。


 \subsubsection {独自実装等のAPI}
API開発者が既存のAPIを利用せずに独自に実装したAPIや、先述2つに含まれないものを独自実装等のAPIと本論文では呼ぶこととする。

\subsection{APIの分析}
河合による先行研究では、Android Developers のAPI リファレンスに記載されているAPIから、暗号・セキュリティに関
するパッケージ、クラス、メソッドを抽出しリスト化を行った。
このリストをもとにAPKにおいてど
れほど暗号技術が利用されているかの分析を行った。

独自実装等のAPIはドキュメントが公開されている可能性
が低いためAPIのリスト化が困難である。これは、
RSAやECC、Cryptoといった暗号、セキュリティ
に関するキーワードをAPIのリストの代わりとし検
索する必要があるためAPKの網羅的調査を行う上
で困難である。

比較して、サードパーティー製API
ではドキュメントが公開されているものもあるので
リスト化の困難性が少ない。
サードパーティ製のAPIの分析ではまずAPIのリスト化を行う必要があるが、サードパーティー製
APIは公式APIとは違いドキュメントが作成され
ていないものがある。存在しない場合はサードパー
ティ製の APIのソースコードを解析し、APIのド
キュメントを作成してからAPIのリストの作成を行う。


そこで、本研究では特
にドキュメントが作成されているサードパーティー製APIを分析対象とする。
ドキュメントが作成されていないサードパーティ製APIと、独自実装等のAPIは今後の課題とする。

\subsection{サードパーティ製API}



\subsection{}
本研究で調査対象であるTinkの分析を行う。
\subsubsection{Tinkの分析}
Tinkは、Googleの暗号技術者とセキュリティエンジニアのグループが開発した、
多言語でクロスプラットフォームな暗号ライブラリである。
Tinkは現在、それぞれのプリミティブを使って実装された、4つの暗号化操作を提供している。
\begin{itemize}
\item 関連データを備えた認証付き暗号(プリミティブ: AEAD)
\item メッセージ認証コード (プリミティブ: MAC)
\item ディジタル署名(プリミティブ: PublicKeySignとPublicKeyVerify)
\item ハイブリッド暗号化(プリミティブ: HybridEncryptとHybridDecrypt)
\end{itemize}
プリミティブとは、単純あるいは基本的な構造や要素のことを言う。
Tinkには、ドキュメントが存在するので、ドキュメントが存在しない
サードパーティ製APIよりリスト化の困難性が少ない。

\subsection{APIの取得方法}
Tinkのドキュメント\cite{Tink Cryptography}からAPIを抽出する。そのクラス(計167個)が持つ
メソッド計000個のリスト化を行った。このリストは、2020年00月のものである。
リストの1部を抜粋し、表0に示す。リスト全体は付録Aに示す。


