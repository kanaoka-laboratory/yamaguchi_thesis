\newpage
\section{まとめ}

近年開発者向けユーザブルセキュリティ、ユーザブルプライバシーの研究分野においてソフトウェア開発者の暗号技術の利用に関する研究が活発になっており、
開発者による暗号技術の利用が適切にされていないことが判明している。そこで本研究ではソフトウェアの暗号技術の利用状況の網羅的な調査の一環として、
Javaで開発されたAndroidアプリケーションを調査対象とし、
Android アプリケーションの現状はどのような暗号技術が利用されているのかまた暗号技術の利用には何らかの傾向があるのかなどを明らかにすることを目的とし調査を行った。
Google社のサードパーティ製APIであるTinkから暗号に関するメソッドを抽出し、705個のメソッドリストを作成した。
307,587個のAPKから展開されたsmaliファイルとこのメソッドリストをマッチングさせることで暗号技術利用の現状を明らかにした。
Tinkを利用しているAPKはAndroidKeysetManagerクラスとそのメソッドが使用されている''mobi.zapzap''だけであった。
このアプリケーション名は、”ZapZap - Mobile Wallet”であり日本ではサービスしていないAndroid版モバイルアプリケーションであった。
使用されているAPIの分析から設定値を保存するSharedPreferencesへのアクセスに
Tink上のAndroidKeysetManagerクラスとそのメソッドを使用しているため、暗号技術そのものとしてTinkは使用されていないことがわかった。
%Tink 利用がAndroid公式APIよりも大幅に少ない理由として、Tink1.0.0のリリース開始が2017年9月と最近である点と、
%Androidでは公式API の利用が中心的である点が考えられる。
今後、より詳しいAPIでの暗号技術利用傾向を知るために他のサードパーティ製APIや、独自実装等のAPIに調査の幅を広げる必要があると考察した。