\newpage
\section{前提知識}

\subsection{Android}
Androidとは、Googel社が2007年に開発した、スマートフォンやタブレット端末など携帯情報機器向けのOperating Systemのこと。主にスマートフォンのOSとして広く普及しており、世界的にApple社の携帯機器向けiOSと市場を2分している。

\subsection{Operating System}
Operating System(以後OS)とは、ソフトウェアの種類の一つで、機器の基本的な管理や制御のための機能や、多くのソフトウェアが共通して利用する基本的な機能などを実装した、システム全体を管理するソフトウェアのこと。

\subsection{Androidアプリケーション}
Javaというプログラミンング言語で作成されている。
Javaプログラムをコンパイルして機械語に変換し、画像などのリソースと合わせてapkというパッケージにすることでAndroidにインストールすることのできるアプリになります。

\subsection{APK}
Android Application Packageの略であり、Android向けのアプリケーションをAndroid端末にインストールできる形式にパッケージにしたもの、もしくはそのファイルのこと。APKとは「.apk」という拡張子を持つAndroidアプリケーションの本体ファイルのことである。
ただし、apkファイル自体はZIP形式で圧縮されており、その中にはアプリケーションの動作に必要なさまざまなファイルが納められている。
apkのファイル形式でなければAndroid端末にアプリケーションをインストールすることができないため、Google Playなどで配信されているアプリケーションは、すべてapkファイルとして公開されている。

\subsetion{smaliファイル}
Androidのapkをapktoolやbaksmaliを用いて逆コンパイルするとsmaliファイルがでてくる。
\subsection{暗号API}
\begin{itemize}
\item MD5
MD5(Message Digest algorithm 5)
MD5は、ハッシュ値を計算するためのハッシュ関数のひとつで、RSA暗号の開発者のひとり、ロン・リベスト氏らによって開発された。
IPsecや、POP before SMTPなど、さまざまなセキュリティープロトコルで使われている一方、最近になって脆弱性も指摘されている。

\item SHA-1
SHA-1(シャーワン)とは Secure Hash Algorithm 1の略で、入力データを一定の手順で計算を行い、
入力値のデータの長さに関わらず決まった長さの文字列を出力するハッシュ関数の一つ。生成された値は「ハッシュ値」(hash value)と呼ばれる。
SHA-1はNSA(米国家安全保障局)が考案し、1995年にNIST(米国標準技術局)によって連邦情報処理標準の一つ(FIPS 180-1)として標準化された。
2005年頃から効率的に攻撃する手法がいくつか発見され十分な安全性が保たれなくなったため、近年では2001年に制定された後継のSHA-2規格への移行が進んでいる。

\item SHA-2
SHA-2はハッシュ関数の計算手順(アルゴリズム)を定義しており、どんな長さのデータからも常に同じ長さのハッシュ値を生成する。
同じ原文からは必ず同じ値が得られる一方、少しでも異なる原文からはまったく違う値が得られる。
データの伝送や複製を行なう際に、入力側と出力側でハッシュ値を求め一致すれば、途中で改ざんや欠落などが起こっていないことを確認することができる。
また、暗号や認証、デジタル署名などの要素技術として様々な場面で利用されている。 
\end{itemize}

\subsection{API}
APIとはApplication Programming Interfaceの略であり、あるコンピュータプログラム(ソフトウェア)の機能や管理するデータなどを、
外部の他のプログラムから呼び出して利用するための手順やデータ形式などを定めた規約のこと。
\begin{itemize}

\item 公式API
Androidの開発者向け公式WebサイトであるAndroid DevelopersのAPIリファレンスに記載されているAPIのこと。

\item サードパーティー製API
サードパーティーとは、特定のハードウェア、OS、ソフトウェア、あるいはサービスなどを対象として、それに対応する(プラットフォーム上で動作する、もしくは互換性のある)製品を販売・提供していること。
企業が提供しているものや, 開発者が提供しているものがあるサードパーティー製APIという。
Google社のtink(ティンク)やFacebook社のConcealn(コンシール)がある。


\item 独自実装等のAPI
API開発者が既存のAPIを利用せずに独自に実装したAPIや,先述2つに含まれないものを独自実装等のAPIとします.

\end{itemize}

\subsection{Android Developers}
公式ドキュメントといった場合Android Developersを指す。
Android Developers とは,Android アプリケーション開発者向けのAndroid公式Webサイトのこと。Androidの詳細やドキュメントが提供されている

\subsection{APIドキュメント}
APIリファレンスとも呼ばれる。
APIによる開発方法やクラス内のメソッドの使用方法を解説した説明書のこと

\subsection{Liuxとは}
LinuxとはOSの一種で、パソコンを動かすのに必要な基本ソフトウェアの1つ。

\subsection{Ubuntu}
UbuntuはLinux系のOS。

\subsection{シェル}
コンピュータのOSを構成するソフトウェアの1つで、利用者からの操作の受付や、利用者への情報の提示などを担当するもの。

\subsection{シェルスプリクト}
OSを操作するためのシェル上で実行(スプリクト言語)。また、そのような言語によって書かれた、複数のOSコマンドや制御文などを組み合わせた簡易なプログラム。一般的にはUNIX系OSのシェルで実行できるものを指す

