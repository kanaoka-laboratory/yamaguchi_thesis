\newpage
\section{前提知識}

\subsection{Android}
Androidとは、Googel社が2007年に開発した、スマートフォンやタブレット端末など携帯情報機器向けのOperating Systemのこと、またAndroid OSが搭載された端末のことである。主にスマートフォンのOSとして広く普及しており、世界的にApple社の携帯機器向けiOSと市場を2分している。

\subsection{Operating System}
Operating System(以後OS)とは、ソフトウェアの種類の一つで、機器の基本的な管理や制御のための機能や、多くのソフトウェアが共通して利用する基本的な機能などを実装した、システム全体を管理するソフトウェアのことである。

\subsection{Androidアプリケーション}
Javaというプログラミンング言語で作成されている。
Javaプログラムをコンパイルして機械語に変換し、画像などのリソースと合わせてapkというパッケージにすることでAndroidにインストールできるアプリである。

\subsection{APK}
Android Application Packageの略であり、Android向けのアプリケーションをAndroid端末にインストールできる形式にパッケージにしたもの、もしくはそのファイルである。
APKとは”.apk”という拡張子を持つが、apkファイル自体はzip形式で圧縮されており、その中にはアプリケーションの動作に必要なさまざまなファイルが納められている。
apkのファイル形式でなければAndroid端末にアプリケーションをインストールすることができないため、Google Playなどで配信されているアプリケーションは、すべてapkファイルとして公開されている。

\subsection{APKストア}
APKストアとは、Androidアプリケーション開発者の作成したAndroidアプリケーションの配信を代行するサービス、およびそれを行っているWebサイトのことである。Androidの公式APKストアは、Android の公式APKストアであるGooglePlay[10]1つのみであり,非公式のAPKストアは数多く存在する。

\subsection{Android Developers}
Android Developers とは,Android アプリケーション開発者向けのAndroid公式Webサイトのことである。Androidの詳細やドキュメントが提供されている。
公式ドキュメントといった場合Android Developersを指す。

\subsection{smaliファイル}
 smaliとは、AndroidのDalvik仮想マシンで使用されるアセンブリ言語Smaliで書かれた開発者ファイルである。
通常、Androidアプリケーションに含まれている実行可能ファイルである。
DEX(Dalvik実行可能)ファイル(.apkファイル)を逆コンパイルすることによって作成される。
ApktoolやBaksmaliを使ってappファイルを展開するとプログラム部分のclasses.dexが展開されsmaliファイルに分かれる。
つまり、smaliファイルとはJavaをコンパイルした後の機械語の状態のファイルである。

\subsection{暗号技術}

\subsubsection {MD5}
MD5は、Message Digest algorithm 5の略であり、ハッシュ値を計算するためのハッシュ関数のひとつで、RSA暗号の開発者のひとり、ロン・リベスト氏らによって開発された。IPsecや、POP before SMTPなど、さまざまなセキュリティープロトコルで使われている一方、最近になって脆弱性も指摘されている。

\subsubsection {SHA-1}
SHA-1(シャーワン)とは Secure Hash Algorithm 1の略で、入力データを一定の手順で計算を行い、入力値のデータの長さに関わらず決まった長さの文字列を出力するハッシュ関数の一つ。生成された値は「ハッシュ値」(hash value)と呼ばれる。
SHA-1はNSA(米国家安全保障局)が考案し、1995年にNIST(米国標準技術局)によって連邦情報処理標準の一つ(FIPS 180-1)として標準化された。
2005年頃から効率的に攻撃する手法がいくつか発見され十分な安全性が保たれなくなったため、近年では2001年に制定された後継のSHA-2規格への移行が進んでいる。

\subsubsection{SHA-2}
 SHA-2はハッシュ関数の計算手順(アルゴリズム)を定義しており、どんな長さのデータからも常に同じ長さのハッシュ値を生成する。
同じ原文からは必ず同じ値が得られる一方、少しでも異なる原文からはまったく違う値が得られる。
データの伝送や複製を行なう際に、入力側と出力側でハッシュ値を求め一致すれば、途中で改竄や欠落などが起こっていないことを確認することができる。
また、暗号や認証、デジタル署名などの要素技術として様々な場面で利用されている。 


\subsection{API}
APIとはApplication Programming Interfaceの略であり、あるコンピュータプログラム(ソフトウェア)の機能や管理するデータなどを、外部の他のプログラムから呼び出して利用するための手順やデータ形式などを定めた規約である。
これは、3種類に分けられる。

\subsubsection {公式API}
Androidの開発者向け公式WebサイトであるAndroid DevelopersのAPIリファレンスに記載されているAPIである。

\subsubsection {サードパーティー製API}
サードパーティーとは、特定のハードウェア、OS、ソフトウェア、あるいはサービスなどを対象として、それに対応する(プラットフォーム上で動作する、もしくは互換性のある)製品を販売・提供しているという意味である。
企業が提供しているものや, 開発者が提供しているものがあるサードパーティー製APIという。
Google社のtink(ティンク)やFacebook社のConcealn(コンシール)がある。


\subsubsection {独自実装等のAPI}
API開発者が既存のAPIを利用せずに独自に実装したAPIや,先述2つに含まれないものを独自実装等のAPIとする。

\subsection{APIドキュメント}
APIリファレンスとも呼ばれる。
APIによる開発方法やクラス内のメソッドの使用方法を解説した説明書である。

\subsection{Liux}
LinuxとはOSの一種で、パソコンを動かすのに必要な基本ソフトウェアの1つ。

\subsection{Ubuntu}
UbuntuはLinux系のOS。

\subsection{シェル}
コンピュータのOSを構成するソフトウェアの1つで、利用者からの操作の受付や、利用者への情報の提示などを担当するもの。

\subsection{シェルスプリクト}
OSを操作するためのシェル上で実行(スプリクト言語)。また、そのような言語によって書かれた、複数のOSコマンドや制御文などを組み合わせた簡易なプログラム。一般的にはUNIX系OSのシェルで実行できるものを指す。



























