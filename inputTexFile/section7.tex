%%%%%%%%%%%%%%%%%%%%今後の課題%%%%%%%%%%%%%%%%%%%%
\newpage
\section{今後の課題}
%%%%%%%%%%%%%%%%%%%%
\subsection{他のサードパーティ製APIの調査}
\ref{sec:third_party}	で述べた通り、サードパーティ製APIはTink以外も存在するので、調査の幅を広げることが可能であると考える。また、他のサードパーティ製APIには\ref{sec:APIの分析}で述べた通り、公式APIとは違い、ドキュメントが公開されていないものもある。サードパーティ製のAPIの分析ではまず、そのAPIについてのドキュメントが存在しているかの確認を行い、存在する場合はそのドキュメントからAPIのリストを作成する。
存在しない場合はサードパーティ製のAPIのソースコードを解析し、APIのドキュメントを作成してからAPIのリストの作成を行う。
APKの解析の前にAPIの解析を行う必要があるという点が公式APIとの違いである。
作成したリストをもとに\ref{sec:調査}と同様の調査を行うことが可能であると考える。
%%%%%%%%%%%%%%%%%%%%
\subsection{独自実装等のAPIの調査}
\ref{sec:APIの分析}で述べた通り、独自実装等のAPIは、ドキュメントが作成され、かつ公開されている可能性が低いため、\ref{sec:APIの取得方法}の様にAPIのドキュメントからAPIのリストを作成することができない。
APIのリストではなく開発者が独自実装等の際に利用する可能性の高いキーワード等を調査、整理しsmaliファイル内で調査する。
そしてその調査対象のキーワード群がどのメソッド名や引数として利用されているのかを調査しさらにそれらを利用しているメソッドやクラスを発見することで独自実装等のAPIを調査することができると考える。