\documentclass[a4j]{jarticle}
\date{}
\usepackage[dvipdfmx]{graphicx}

\begin{document}


%表題
\makeatletter %ここから\makeatotherまで触らなくていい
	\def\@thesis{令和2年度 東邦大学理学部情報科学科 卒業研究}
	\def\id#1{\def\@id{#1}}
	\def\department#1{\def\@department{#1}}
	
	\def\@maketitle{
		\begin{center}
			\vspace{10mm}
			{\large \@thesis \par}	%修士論文と記載される部分
			\vspace{50mm}
			{\huge\bf \@title \par}	% 論文のタイトル部分
			\vspace{15mm}
			{\Large 学籍番号 \@id \par}	% 学籍番号部分
			\vspace{5mm}
			{\Large \@author \par}	% 氏名
			\vspace{50mm}
		\end{center}
		\begin{flushright}
			{\large 金岡研究室}
		\end{flushright}
	}
\makeatother

\title{Androidアプリケーションにおけるサードパーティー製APIでの暗号技術利用動向の調査} %中括弧の中に卒論のタイトル
\id{5517097} %学籍番号
\author{山口 千尋} %名前
\maketitle{\title} %表紙の出力
\thispagestyle{empty} %このページ(表紙)のページ番号を消去
\newpage %強制改ページ

\tableofcontents %目次の出力
\section{はじめに}
\section{前提知識}
\section{関連研究}
\section{提案手法のメイン部分}
\section{}
\section{}
\section{}
\subsection{}


\section{}
\subsection{}


・前提知識
 - Android
 - Android OS
 - APK
 - Androidアプリケーション
 - API
  ・公式API
  ・サードパーティー製API
  ・独自実装等のAPI
 - Android Developers
 - APIリファレンス
 - Liuxとは
 - Ubuntu
 - シェル
 - シェルスプリクト

関連研究
 - 河合さんの研究
 - 

提案手法のメイン部分
 - *API取得方法
   GoogleのTinkの種類
    関連データを備えた認証付き暗号(プリミティブ: AEAD)
    メッセージ認証コード (プリミティブ: MAC)
    ディジタル署名(プリミティブ: PublicKeySignとPublicKeyVerify)
    ハイブリッド暗号化(プリミティブ: HybridEncryptとHybridDecrypt)
 - *smaliファイル取得方法
 - Ubuntu設定方法
 - シェルの作成方法
   シェルスプリクトの解説
 
提案手法の試作
 - Ubuntuでの実行のやり方
   使用したコマンド類

実験の結果
 - 利用されているAPIの調査結果
   

今後の課題(残課題)
 - Google以外のサードパーティー製APIの調査
 - 独自実装等のAPIの調査
 - 

まとめ

参考文献
Mobile Operating System Market Share(AndroidOS普及率)
Android Developers. 
河合さん論文




\newpage
\section{はじめに} 

\newpage
\section{前提知識}
\subsection{Android} 
Androidとは、Googel社が2007年に開発した、スマートフォンやタブレット端末など携帯情報機器向けのOperating Systemのこと。主にスマートフォンのOSとして広く普及しており、世界的にApple社の携帯機器向けiOSと市場を2分している。

\begin{itemize}
\item まずは
\item この章のなかで書くことを
\item 箇条書きで書き出してみる
\item ことから始めましょう
\end{itemize}

\subsection{前提知識B}
あああああ

あああああ

\begin{itemize}
\item まずは
\item この章のなかで書くことを
\item 箇条書きで書き出してみる
\item ことから始めましょう
\end{itemize}


\newpage
\section{関連研究} %先に書く
\subsection{Xに関連した研究}
あああああ

あああああ

\begin{itemize}
\item まずは
\item この章のなかで書くことを
\item 箇条書きで書き出してみる
\item ことから始めましょう
\end{itemize}

\subsection{Yに関連した研究}
あああああ

あああああ

\begin{itemize}
\item まずは
\item この章のなかで書くことを
\item 箇条書きで書き出してみる
\item ことから始めましょう
\end{itemize}


\newpage
\section{提案手法のメインな部分}
あああああ

あああああ

\begin{itemize}
\item まずは
\item この章のなかで書くことを
\item 箇条書きで書き出してみる
\item ことから始めましょう
\end{itemize}


\newpage
\section{提案手法の試作みたいなのを書く部分}
あああああ

あああああ

\begin{itemize}
\item まずは
\item この章のなかで書くことを
\item 箇条書きで書き出してみる
\item ことから始めましょう
\end{itemize}

\newpage
\section{試作を用いて評価}
あああああ

あああああ

\begin{itemize}
\item まずは
\item この章のなかで書くことを
\item 箇条書きで書き出してみる
\item ことから始めましょう
\end{itemize}

\newpage
\section{まとめ}
あああああ

あああああ

\begin{itemize}
\item まずは
\item この章のなかで書くことを
\item 箇条書きで書き出してみる
\item ことから始めましょう
\end{itemize}



\newpage
\begin{thebibliography}{99}
\bibitem{bib01}
だれだれ, "文献1", 年度
\bibitem{bib02}
だれだれ, "文献2", 年度
\bibitem{bib03}
だれだれ, "文献3", 年度
\bibitem{bib04}
だれだれ, "文献4", 年度
\bibitem{bib05}
だれだれ, "文献5", 年度
\bibitem{bib06}
だれだれ, "文献6", 年度
\bibitem{bib07}
だれだれ, "文献7", 年度
\bibitem{bib08}
だれだれ, "文献8", 年度
\bibitem{bib09}
だれだれ, "文献9", 年度
\bibitem{bib010}
だれだれ, "文献10", 年度

\end{thebibliography}

\end{document}